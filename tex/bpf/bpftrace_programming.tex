\subsection{Пример программы на bpftrace}

Программа измеряет время, затраченное на выполнение функции ядра vfs\_read(): \\

\cn{\#!/usr/local/bin/bpftrace} \\
\cn{kprobe:vfs\_read }
\cn{\{ }
\ci{	@start[tid] = nsecs; }
\cn{\}} \\
\cn{kretprobe:vfs\_read }
\cn{/@start[tid]/ }
\cn{\{ }
\ci{	\$duration\_us = (nsecs - @start[tid]) / 1000; }
\ci{	@us = hist(\$duration\_us); }
\ci{	delete(@start[tid]); }
\cn{\} }

\subsection{Развернутые циклы}
\ci{unroll (count) \{ statements \}}
Аргумент count — это целочисленный литерал (константа) с максимально возможным значением 20. \\


\subsection{Встроенные переменные}
Встроенные переменные предопределены в bpftrace и обычно доступны только для
чтения. \\
Наиболее важные встроенные переменные в bpftrace:

\begin{tabular}{|l|c|p{8cm}|}
\hline
\rowcolor{gray!40}
Переменная & Тип & Описание \\
\hline
pid & int & Идентификатор процесса (tgid в ядре) \\
\hline
tid & int & Идентификатор потока (pid в ядре) \\
\hline
uid & int & Идентификатор пользователя \\
\hline
username & string & Имя пользователя \\
\hline
nsecs & int & Отметка времени в наносекундах \\
\hline
elapsed & int & Время в наносекундах, прошедшее с начала инициализации bpftrace \\
\hline
cpu & int & Идентификатор процессора \\
\hline
comm & string & Имя процесса \\
\hline
kstack & string & Трассировка стека в пространстве ядра \\
\hline
ustack & string & Трассировка стека в пространстве пользователя \\
\hline
arg0, ..., argN & int & Аргументы зондов некоторых типов \\
\hline
args & struct & Аргументы зондов некоторых типов \\
\hline
retval & int & Возвращаемые значения для зондов некоторых типов \\
\hline
func & string & Имя трассируемой функции \\
\hline
probe & string & Полное имя текущего зонда \\
\hline
curtask & int & task\_struct в ядре как 64-битное целое без знака (допускается приведение типа) \\
\hline
cgroup & int & Идентификатор сgroup  \\
\hline
\$1, ..., \$N & \makecell[c]{int \\ char*} & Позиционные параметры программы bpftrace \\
\hline
\end{tabular}


\subsection{Карты}
Формат определения: \\
\ci{@name}
\ci{@name[key]}
\ci{@name[key1, key2[, ...]]}
Примеры: \\
\ci{@start = nsecs;}
\ci{@last[tid] = nsecs;}
\ci{@bytes = hist(retval);}
\ci{@who[pid, comm] = count();}

\subsection{Наиболее важные функции bpftrace}
\begin{tabular}{|l|p{9cm}|}
\hline
\rowcolor{gray!40}
Функция & Описание \\
\hline
\texttt{printf(char *fmt [, ...])} & Форматированный вывод \\
\hline
\texttt{time(char *fmt)} & Форматированный вывод времени \\
\hline
\texttt{join(char *arr[])} & Выводит массив строк, объединяя их через пробел \\
\hline
\texttt{str(char *s [, int len])} & Возвращает строку, на которую ссылается указатель s, с необязательным ограничителем длины len  \\
\hline
\texttt{kstack(int limit) } & Возвращает трассировку стека ядра с глубиной до limit  \\
\hline
\texttt{ustack(int limit) } & Возвращает трассировку стека в пространстве пользователя с глубиной до limit \\
\hline
\texttt{ksym(void *p) } & Определяет символ по адресу в пространстве ядра и возвращает строку с ним  \\
\hline
\texttt{usym(void *p) } & Определяет символ по адресу в пространстве пользователя и возвращает строку с ним  \\
\hline
\texttt{kaddr(char *name) } & Определяет адрес символа в пространстве ядра  \\
\hline
\texttt{uaddr(char *name) } & Определяет адрес символа в пространстве пользователя  \\
\hline
\texttt{reg(char *name) } & Возвращает значение, хранящееся в указанном регистре  \\
\hline
\texttt{ntop([int af,] int addr) } & Возвращает строковое представление IP-адреса  \\
\hline
\texttt{system(char *fmt [, ...]) } & Выполняет команду в командной оболочке  \\
\hline
\texttt{cat(char *filename) } & Выводит содержимое указанного файла  \\
\hline
\texttt{exit() } & Завершает выполнение bpftrace  \\
\hline
\end{tabular}

\subsection{Наиболее важные функции-карты в bpftrace}

\begin{tabular}{|l|p{9cm}|}
\hline
\rowcolor{gray!40}
Функция & Описание \\
\hline
\texttt{count()} & Подсчитывает число вхождений \\
\hline
\texttt{sum(int n)} & Подсчитывает сумму значений \\
\hline
\texttt{avg(int n)} & Вычисляет среднее значение \\
\hline
\texttt{min(int n)} & Запоминает минимальное значение \\
\hline
\texttt{max(int n)} & Запоминает максимальное значение \\
\hline
\texttt{stats(int n)} & Возвращает общее количество, среднее и сумму \\
\hline
\texttt{hist(int n)} & Выводит гистограмму значений с шагом, равным степени двойки \\
\hline
\makecell[l]{\texttt{lhist(int n, int min,} \\ \texttt{\ \ int max, int step)} }  & Выводит линейную гистограмму значений  \\
\hline
\texttt{delete(@m[key])} & Удаляет пару ключ — значение из карты  \\
\hline
\texttt{print(@m [, top [, div]])} & Выводит содержимое карты с необязательными ограничением на вывод определенного числа наибольших значений и  делителем   \\
\hline
\texttt{clear(@m)} & Удаляет все пары ключ — значение из карты  \\
\hline
\texttt{zero(@m)} & Сбрасывает все значения в карте в ноль  \\
\hline
\end{tabular}


\noindent Вывод частоты вхождений в течение каждого интервала: \\
\cn{\# bpftrace -e 'tracepoint:block:block\_rq\_i* \{@[probe] = count();\}}
\ci{\indent interval:s:1 \{ print(@); clear(@); \}' }

\noindent Подсчет общего числа байтов, прочитанных системным вызовом read(2): \\
\cn{\# bpftrace -e 'tracepoint:syscalls:sys\_exit\_read /args->ret>0/ \{}
\ci{\indent	@bytes = sum(args->ret); \}'}

\noindent Гистограммы размеров блоков, успешно прочитанных системным вызовом read(2): \\
\cn{\# bpftrace -e 'tracepoint:syscalls:sys\_exit\_read \{@ret = hist(args->ret);\}'}


