\section{Зонды}

Поиск зондов: \\
\ci{\# bpftrace -l 'kprobe:vfs\_*'}
Подробная информация о зонде: \\
\ci{\# bpftrace -lv tracepoint:syscalls:sys\_enter\_accept}


\textbf{Типы зондов в bpftrace:} \\
\begin{tabular}{|l|c|p{12cm}|}
\hline
\rowcolor{gray!40}
\multicolumn{2}{|c|}{Тип (Псевдоним)} & Описание \\
\hline
tracepoint & t & Инструментируют статические точки трассировки в ядре  \\
\hline
usdt & U & Инструментируют статические точки трассировки в пространстве пользователя  \\
\hline
kprobe & k & Инструментируют динамические точки вызовов функций ядра  \\
\hline
kretprobe & kr & Инструментируют динамические точки возврата из функций ядра  \\
\hline
uprobe & u & Инструментируют динамические точки вызовов функций в пространстве пользователя  \\
\hline
uretprobe & ur & Инструментируют динамические точки возврата из функций в пространстве пользователя  \\
\hline
software & s & Программные события в пространстве ядра  \\
\hline
hardware & h & Инструментируют аппаратные счетчики  \\
\hline
profile & p & Производят выборку по времени для всех процессоров  \\
\hline
interval & i & Производят выборку в течение интервала (для одного процессора)  \\
\hline
BEGIN & & Запуск bpftrace  \\
\hline
END & & Завершение bpftrace  \\
\hline
\end{tabular}

\subsection{tracepoint}
Зонды типа tracepoint инструментируют статические точки трассировки в ядре.
Формат определения: \\
\ci{tracepoint:tracepoint\_name}

\noindent Страница справочного руководства: \\
\ci{ssize\_t read(int fd, void *buf, size\_t count); }
В точке трассировки sys\_enter\_read эти аргументы должны быть доступны как
args-> fd, args-> buf и args->count. \\

\cn{\# bpftrace -lv tracepoint:syscalls:sys\_enter\_read}
\cn{tracepoint:syscalls:sys\_enter\_read}
\ci{int \_\_syscall\_nr; }
\ci{unsigned int fd; }
\ci{char * buf; }
\ci{size\_t count; }

\subsection{usdt}
Зонды этого типа инструментируют статические точки трассировки в пространстве
пользователя. Формат определения: \\
\ci{usdt:binary\_path:probe\_name}
\ci{usdt:library\_path:probe\_name}
\ci{usdt:binary\_path:probe\_namespace:probe\_name}
\ci{usdt:library\_path:probe\_namespace:probe\_name}

\noindent Получить список зондов, доступных в файле, можно с помощью параметра \texttt{-l}, например: \\
\cn{\# bpftrace -l 'usdt:/usr/local/cpython/python' }
\ci{usdt:/usr/local/cpython/python:line }
\ci{usdt:/usr/local/cpython/python:function\_\_entry }
\ci{usdt:/usr/local/cpython/python:function\_\_return }
\ci{usdt:/usr/local/cpython/python:import\_\_find\_\_load\_\_start }
\ci{usdt:/usr/local/cpython/python:import\_\_find\_\_load\_\_done }
\ci{usdt:/usr/local/cpython/python:gc\_\_start }
\ci{usdt:/sur/local/cpython/python:gc\_\_done }
Можно получить и список зондов USDT в выполняющемся процессе, в этом случае
вместо имени файла следует использовать параметр \texttt{-p PID}.

\subsection{kprobe и kretprobe}
Зонды этого типа используются для динамической инструментации ядра. Формат
определения: \\
\ci{kprobe:function\_name}
\ci{kretprobe:function\_name}

Зонды kprobe имеют аргументы arg0, arg1, ..., argN — входные аргументы функции,
как целые 64-битные целые без знака. Если какой-то из них является указателем
на структуру языка C, его можно привести к типу этой структуры.

Единственный аргумент kretprobe - встроенная переменная retval - содержит 
возвращаемое значение функции. retval всегда имеет тип uint64.

\subsection{uprobe и uretprobe}
Зонды этого типа используются для динамической инструментации кода в пространстве пользователя. Формат определения: \\
\ci{uprobe:binary\_path:function\_name}
\ci{uprobe:library\_path:function\_name}
\ci{uretprobe:binary\_path:function\_name}
\ci{uretprobe:library\_path:function\_name}

\subsection{software и hardware}
Зонды этого типа инструментируют предопределенные программные (software)
и аппаратные (hardware) события. Формат определения: \\
\ci{software:event\_name:count}
\ci{software:event\_name:}
\ci{hardware:event\_name:count}
\ci{hardware:event\_name:}

\textbf{Программные события:} \\
\begin{tabular}{|l|c|c|p{8cm}|}
\hline
\rowcolor{gray!40}
\multicolumn{3}{|c|}{\makecell{Имя события / Псевдоним \\ Значение счетчика по умолчанию}} & Описание \\
\hline
cpu-clock & cpu & $10^6$ & Фактическое время процессора \\
\hline
task-clock & & $10^6$ & Время использования процессора задачей 
  (увеличивается, только когда задача выполняется на процессоре) \\
\hline
page-faults & faults & 100 & Отказы страниц \\
\hline
context-switches & cs & 1000 & Переключение контекста \\
\hline
cpu-migrations & & 1 & Миграция потоков процессора \\
\hline
minor-faults & & 100 & Незначительные отказы страниц: устранены за счет памяти \\
\hline
major-faults & & 1 & Значительные отказы страниц: устранены за счет
ввода/вывода из хранилища \\
\hline
alignment-faults & & 1 & Ошибки выравнивания \\
\hline
emulation-faults & & 1 & Ошибки эмуляции \\ 
\hline
dummy & & 1 & Искусственное событие для тестирования \\
\hline
bpf-output & & 1 & Канал вывода BPF \\
\hline
\end{tabular}
\\

\textbf{Аппаратные события:} \\
\begin{tabular}{|l|c|c|p{8cm}|}
\hline
\rowcolor{gray!40}
\multicolumn{3}{|c|}{\makecell{Имя события / Псевдоним \\ Значение счетчика по умолчанию}} & Описание \\
\hline

cpu-cycles & cycles & $10^6$ & Такты процессора \\
instructions & & $10^6$ & Инструкции процессора \\
\hline
cache-references & & $10^6$ & Обращения к кэшу процессора последнего уровня \\
\hline
cache-misses & & $10^6$ & Промахи кэша процессора последнего уровня \\
\hline
branch-instructions & branches & $10^5$ & Инструкции ветвления \\
\hline
bus-cycles & & $10^5$ & Такты шины \\
\hline
frontend-stalls & & $10^6$ & Пропуски циклов внешнего интерфейса процессора 
(например, на время выборки инструкций) \\
\hline
backend-stalls & & $10^6$ & Пропуски внутренних циклов процессора
(например, на время загрузки/сохранения данных) \\
\hline
ref-cycles & & $10^6$ & Циклы обращения к процессору (не масштабируется в турборежиме) \\
\hline

\end{tabular}


\subsection{profile и interval}
Зонды этого типа инструментируют события, имеющие отношение к времени. Формат определения: \\
\ci{profile:hz:rate}
\ci{profile:s:rate}
\ci{profile:ms:rate}
\ci{profile:us:rate}
\ci{interval:s:rate}
\ci{interval:ms:rate}

Зонды profile срабатывают для всех процессоров, interval только для одного процессора.
Во втором поле можно указать:
\begin{itemize}
	\item hz: герцы (событий в секунду);
	\item s: секунды;
	\item ms: миллисекунды;
	\item us: микросекунды.
\end{itemize}

