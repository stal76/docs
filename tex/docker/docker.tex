\documentclass[12pt, a4paper]{article}
\usepackage[utf8]{inputenc}
\usepackage[russian]{babel}
\usepackage{cmap}
\usepackage{amsmath}
\usepackage{amsfonts}
\usepackage{amssymb}
\usepackage{makeidx}
\usepackage{graphicx}
\usepackage{hyperref}

\newcommand{\ci}[1]{\indent\texttt{#1} \\}
\newcommand{\cn}[1]{\noindent\texttt{#1} \\}
\newcommand{\cci}[2]{\indent\texttt{#1\indent \#} - #2 \\}
\newcommand{\ccn}[2]{\noindent\texttt{#1\indent \#} - #2 \\}
\newcommand{\tc}[2]{\noindent #2 \\ \indent\texttt{#1} \\}

\begin{document}
	
	\begin{center} {\Huge Docker - справка} \end{center}
	
%	\noindent Составлено по материалам курса Ильи Кантора от JavaScript.ru: \\ 
%	\indent \href{https://www.youtube.com/watch?v=W4hoc24K93E&list=PLDyvV36pndZFHXjXuwA_NywNrVQO0aQqb}
%	{https://www.youtube.com/watch?v=W4hoc24K93E}  \\

	% Основные команды
	\section{Основные команды}

\subsection{Общие}

\noindent Создать пользователя для запуска docker: \\
\ci{\$useradd -m -s /bin/bash <user>}
\ci{\$usermod -uG docker <user>}
\ci{\$su - <user>}

\ccn{\$docker {-}-version}{просмотр версии docker}
\ccn{\$docker version}{подробная информация версии docker}

\ccn{\$service docker status}{проверить статус docker}

\ccn{\$docker system prune -a {-}-volumes}{удалить ВСЕ данные в docker}

\subsection{Образы}

\ccn{\$docker pull <image>}{загрузить образ}
\ccn{\$docker images}{список всех образов}
\ccn{\$docker rmi <image>}{удалить образ}
\ccn{\$docker image inspect <image>}{информация об образе}

\noindent Варианты значений тэгов: latest, <version>, alpine (минимального размера) \\


\subsection{Запуск из образов}

\ccn{\$docker run <image>}{запустить образ, будет загружен если его нет}
\ccn{\$docker run hello-world}{запустить базовый контейнер hello-world}
\ccn{\$docker run <image> <command>}{запустить образ и в нем команду}

\noindent Дополнительные параметры: \\
\begin{tabular}{l l}
\indent \texttt{-d} & запустить в режиме ``detach mode'' \\
\indent \texttt{-it} & запустить интерактивно (i - interactive, t - terminal) \\
\indent \texttt{{-}-rm} & удалить контейнер после завершения \\
\indent \texttt{{-}-name <name>} & задать имя для нового контейнера \\
\end{tabular}

\subsection{Контейнеры}

\ccn{\$docker ps}{список запущенных контейнеров}
\ccn{\$docker ps -a}{список всех контейнеров}
\ccn{\$docker rm <container>}{удалить контейнер}
\ccn{\$docker container prune}{удалить все контейнеры}
\ccn{\$docker start <container>}{запустить контейнер}

\tc{\$docker exec -it <container> <command>}{Выполнить интерактивно команду в контейнере:}

\ccn{\$docker pause container}{поставить на паузу}
\ccn{\$docker unpause container}{возобновить работу}
\ccn{\$docker stop container}{остановить работу}
\ccn{\$docker kill container}{убить}

\ccn{\$docker logs <container>}{Последние log'и контейнера}
\ccn{\$docker logs -f <container>}{life-logs}

\ccn{\$docker inspect <container>}{просмотр подробной информации}
\ccn{\$docker container inspect <container>}{или так}
\ccn{\$docker stats <container>}{статистика использования ресурсов}


\tc{\$docker run -d {-}-rm {-}-name MyContainer ubuntu:20.04 echo ``Hello''}{Пример команды:}


	% Port mapping
	\section{Port Mapping}

\tc{\$docker run -p <port1>:<port2> <container>}
{Все запросы на <port1> сервера пробрасывать в контейнер на порт <port2>:}


	% Environment variables
	\section{Environment variables}

\tc{\$docker run -e <var>=<name>}{Задать в контейнере переменную окружения:}

\ccn{env}{получить все системные переменные окружения}
\ccn{export <var>=<value>}{задать переменнуж окружения}
	
	% Volumes
	\section{Docker Volumes}

\noindent\textbf{Host Volumes} \\
\tc{\$docker run -v <dir\_server>:<dir\_container>[:ro] ...}{Монтировать фиксированный каталог сервера в каталог контейнера:}
\indent\cci{:ro}{read-only}

\noindent\textbf{Anonymous Volumes} \\
\tc{\$docker run -v <dir\_container> ...}{Монтировать каталог сервера в каталог контейнера:}
\tc{/var/lib/docker/volumes/<HASH>/\_data}{Каталог сервера находится в:}

\noindent\textbf{Named Volumes} \\
\tc{\$docker run -v <volume\_name>:<dir\_container> ...}{Монтировать именованный volume в каталог контейнера:}
\tc{/var/lib/docker/volumes/<volume\_name>/\_data}{Каталог сервера находится в:}

\noindent Команды управления volumes: \\
\cci{\$docker volume ls}{Список volumes}
\cci{\$docker volume create <volume\_name>}{Создать named volume}
\cci{\$docker volume rm <volume\_name>}{Удалить volume}
	
	% Network
	\section{Docker Network}

Создается default сеть типа bridge: \\
- Сетевой интерфейс docker0 (172.17.0.1/16) \\
- bridge docker0: 172.17.0.0/16 \\

\noindent Если контейнеры в одной сети (кроме default), то они доступны по dns по именам. \\

\noindent Кроме bridge возможные типы сети: \\
\cci{{-}{-}network=host}{получают адрес хоста}
\cci{{-}{-}network=none}{без сети}

\noindent Также доступны типы сети: \\
\cci{macvlan}{каждый контейнер получает свой mac адрес}
\cci{ipvlan}{один и тот же mac адрес у всех контейнеров}
\cci{overlay}{Docker Swarm Cluster}

\tc{\$docker network create {-}{-}driver bridge <net>}{Создать сеть с типом Bridge:}
\cci{\indent{-}{-}driver host}{с типом host}
\cci{\indent{-}{-}driver none}{без сети}

\tc{\$docker run {-}{-}net <net> ...}{Запустить контейнер в сети:}

\ccn{\$docker network ls}{список сетей}
\ccn{\$docker network inspect <net>}{информация о сети}
\tc{\$docker network create -d bridge {-}{-}subnet 192.168.10.0/24 $\backslash$ \\ \indent\indent {-}{-}gateway 192.168.10.1 <net>}{Пример создания сети с заданными адресами:}
\ccn{\$docker network rm <net>}{удалить сеть}

\tc{nicolaka/netshoot}{Docker образ с установленными сетевыми инструментами:}

\tc{\$docker network connect <net> <container>}{Подключить работающий контейнер к сети:}
\tc{\$docker network disconnect <net> <container>}{Отключить от сети:}

\tc{\$docker network create -d macvlan ${\backslash}$ }{Создать сеть с адресами из того же диапазона, что и сам сервер:}
\indent\ci{{-}{-}subnet 192.168.100.0/24 {-}{-}gateway 192.168.100.1 ${\backslash}$ }
\indent\ci{{-}{-}ip-range 192.168.100.99/32 -o parent=ens18 <net> }

\tc{\$docker run ... {-}{-}ip <ip\_address> {-}{-}net <net> ...}{Присвоить фиксированный ip адрес:}
	
	% Dockerfile
	\section{Создание контейнеров, Dockerfile}

Структура Dockerfile: \\
	\begin{tabular}{| l | l |}
		\hline
		Раздел & Пример \\
		\hline
		Базовый образ & \texttt{FROM ubuntu:22.04} \\
		Описание образа & \texttt{LABEL author=...} \\
		Команды & \texttt{RUN apt update} \\
		  & \texttt{RUN apt install ...} \\
		Рабочие директории & \texttt{WORKDIR <dir>} \\
		Файлы & \texttt{COPY <src> <dst>} \\
		& \texttt{RUN chmod +x} \\
		Указание переменных & \texttt{END var=value} \\
		Порты & \texttt{EXPOSE 80} \\
		Команды при запуске контейнера & \texttt{CMD [``<cmd>'', ``<arg1>'', ...]} \\
		& \texttt{ENTRYPOINT [``<cmd>'', ``<arg1>'', ...]} \\
		\hline
	\end{tabular}
\\

\noindent Пример простого Dockerfile: \\
\ci{FROM UBUNTU:22.04}
\ci{CMD [``echo'', ``Hello'']}
Последнюю строку можно записать: \texttt{CMD ECHO ``Hello''} \\

\tc{\$docker build <dir>}{Создать образ, в каталоге должен быть файл Dockerfile:}
\tc{\$docker build -f <docker\_file>}{Создать образ указав путь к Dockerfile:}
\tc{\$docker tag <image> <name>:<tag>}{Изменить имя и тэг у существующего образа:}
\tc{\$docker build -t <name>:<tag> <dir>}{Создать образ и задать имя и тэг:}

\noindent Отличия команд при запуске: \\
\cci{ENTRYPOINT}{неизменяемая команда}
\cci{CMD}{команда может быть изменена при запуске контейнера}

\noindent Пример Dockerfile для nginx: \\
\ci{FROM UBUNTU:22.04}
\ci{RUN apt-get update}
\ci{RUN apt-get install nginx -y}
\ci{CMD [``nginx'', ``-g'', ``daemon off'']}


\noindent Открытие портов в контейнер (EXPOSE ports: \\
\ci{EXPOSE 80}
\ci{EXPOSE 443/TCP}
\tc{\$docker -P ...}{Пробросить случайные порты сервера в контейнер}

	
	\section{Docker Compose}

Сравнение запуска docker и используя docker-compose.yaml: \\
\begin{tabular}{| l | l |}
	\hline
	Docker CLI & docker-compose.yaml \\
	\hline
	\texttt{docker run} & \texttt{version: "3.5"} \\
	 & \texttt{services:} \\
	 & \texttt{\indent web-server:} \\
	 \texttt{nginx:stable} & \texttt{\indent\indent image: nginx:stable} \\
	 \texttt{{-}-name my\_nginx} & \texttt{\indent\indent container\_name: my\_nginx} \\
	 & \texttt{\indent \indent volumes:} \\
	 \texttt{-v ...:...} & \texttt{\indent \indent \indent - '/opt/web/html:/var/www/html'} \\
	 \texttt{-v ...:...} & \texttt{\indent \indent \indent - '/opt/web/pics:/var/www/pics'} \\
	 & \texttt{\indent \indent environment:} \\
	 \texttt{-e ...} & \texttt{\indent \indent \indent -NGINX\_HOST=web} \\
	 \texttt{-e ...} & \texttt{\indent \indent \indent -NGINX\_PORT=80} \\
	 & \texttt{\indent \indent ports: } \\
	 \texttt{-p 80:80} & \texttt{\indent \indent \indent - "80:80" } \\
	 \texttt{-p 443:443} & \texttt{\indent \indent \indent - "443:443" } \\
	 & \texttt{\indent \indent restart: unless-stopped \# always, on-failure } \\
	 & \texttt{\indent \indent networks: } \\
	 & \texttt{\indent \indent \indent default: } \\
	 & \texttt{\indent \indent \indent \indent driver: bridge } \\
	 \texttt{-net webnet} & \texttt{\indent \indent \indent \indent name: webnet } \\
	\hline
\end{tabular}

\tc{\$docker compose up [-d]}{Запустить через Docker Compose:}
\tc{\$docker compose stop}{Остановить:}
\\
\tc{image: <image\_name>}{Вместо:}
\tc{build .}{Можно собирать из Dockerfile:}

	\section{Docker Portainer}

Удобный web интерфейс для управления Docker. Упрощает жизнь с Docker.
Есть бесплатная версия - Community Edition. \\

\noindent Статья по установке Docker Portainer: \\
\href{https://losst.pro/ustanovka-docker-portainer}{https://losst.pro/ustanovka-docker-portainer}
		
\end{document}
