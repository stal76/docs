\section{Docker Network}

Создается default сеть типа bridge: \\
- Сетевой интерфейс docker0 (172.17.0.1/16) \\
- bridge docker0: 172.17.0.0/16 \\

\noindent Если контейнеры в одной сети (кроме default), то они доступны по dns по именам. \\

\noindent Кроме bridge возможные типы сети: \\
\cci{{-}{-}network=host}{получают адрес хоста}
\cci{{-}{-}network=none}{без сети}

\noindent Также доступны типы сети: \\
\cci{macvlan}{каждый контейнер получает свой mac адрес}
\cci{ipvlan}{один и тот же mac адрес у всех контейнеров}
\cci{overlay}{Docker Swarm Cluster}

\tc{\$docker network create {-}{-}driver bridge <net>}{Создать сеть с типом Bridge:}
\cci{\indent{-}{-}driver host}{с типом host}
\cci{\indent{-}{-}driver none}{без сети}

\tc{\$docker run {-}{-}net <net> ...}{Запустить контейнер в сети:}

\ccn{\$docker network ls}{список сетей}
\ccn{\$docker network inspect <net>}{информация о сети}
\tc{\$docker network create -d bridge {-}{-}subnet 192.168.10.0/24 $\backslash$ \\ \indent\indent {-}{-}gateway 192.168.10.1 <net>}{Пример создания сети с заданными адресами:}
\ccn{\$docker network rm <net>}{удалить сеть}

\tc{nicolaka/netshoot}{Docker образ с установленными сетевыми инструментами:}

\tc{\$docker network connect <net> <container>}{Подключить работающий контейнер к сети:}
\tc{\$docker network disconnect <net> <container>}{Отключить от сети:}

\tc{\$docker network create -d macvlan ${\backslash}$ }{Создать сеть с адресами из того же диапазона, что и сам сервер:}
\indent\ci{{-}{-}subnet 192.168.100.0/24 {-}{-}gateway 192.168.100.1 ${\backslash}$ }
\indent\ci{{-}{-}ip-range 192.168.100.99/32 -o parent=ens18 <net> }

\tc{\$docker run ... {-}{-}ip <ip\_address> {-}{-}net <net> ...}{Присвоить фиксированный ip адрес:}