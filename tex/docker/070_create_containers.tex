\section{Создание контейнеров, Dockerfile}

Структура Dockerfile: \\
	\begin{tabular}{| l | l |}
		\hline
		Раздел & Пример \\
		\hline
		Базовый образ & \texttt{FROM ubuntu:22.04} \\
		Описание образа & \texttt{LABEL author=...} \\
		Команды & \texttt{RUN apt update} \\
		  & \texttt{RUN apt install ...} \\
		Рабочие директории & \texttt{WORKDIR <dir>} \\
		Файлы & \texttt{COPY <src> <dst>} \\
		& \texttt{RUN chmod +x} \\
		Указание переменных & \texttt{END var=value} \\
		Порты & \texttt{EXPOSE 80} \\
		Команды при запуске контейнера & \texttt{CMD [``<cmd>'', ``<arg1>'', ...]} \\
		& \texttt{ENTRYPOINT [``<cmd>'', ``<arg1>'', ...]} \\
		\hline
	\end{tabular}
\\

\noindent Пример простого Dockerfile: \\
\ci{FROM UBUNTU:22.04}
\ci{CMD [``echo'', ``Hello'']}
Последнюю строку можно записать: \texttt{CMD ECHO ``Hello''} \\

\tc{\$docker build <dir>}{Создать образ, в каталоге должен быть файл Dockerfile:}
\tc{\$docker build -f <docker\_file>}{Создать образ указав путь к Dockerfile:}
\tc{\$docker tag <image> <name>:<tag>}{Изменить имя и тэг у существующего образа:}
\tc{\$docker build -t <name>:<tag> <dir>}{Создать образ и задать имя и тэг:}

\noindent Отличия команд при запуске: \\
\cci{ENTRYPOINT}{неизменяемая команда}
\cci{CMD}{команда может быть изменена при запуске контейнера}

\noindent Пример Dockerfile для nginx: \\
\ci{FROM UBUNTU:22.04}
\ci{RUN apt-get update}
\ci{RUN apt-get install nginx -y}
\ci{CMD [``nginx'', ``-g'', ``daemon off'']}


\noindent Открытие портов в контейнер (EXPOSE ports: \\
\ci{EXPOSE 80}
\ci{EXPOSE 443/TCP}
\tc{\$docker -P ...}{Пробросить случайные порты сервера в контейнер}
