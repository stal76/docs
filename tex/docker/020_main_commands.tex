\section{Основные команды}

\subsection{Общие}

\noindent Создать пользователя для запуска docker: \\
\ci{\$useradd -m -s /bin/bash <user>}
\ci{\$usermod -uG docker <user>}
\ci{\$su - <user>}

\ccn{\$docker {-}-version}{просмотр версии docker}
\ccn{\$docker version}{подробная информация версии docker}

\ccn{\$service docker status}{проверить статус docker}

\ccn{\$docker system prune -a {-}-volumes}{удалить ВСЕ данные в docker}

\subsection{Образы}

\ccn{\$docker pull <image>}{загрузить образ}
\ccn{\$docker images}{список всех образов}
\ccn{\$docker rmi <image>}{удалить образ}
\ccn{\$docker image inspect <image>}{информация об образе}

\noindent Варианты значений тэгов: latest, <version>, alpine (минимального размера) \\


\subsection{Запуск из образов}

\ccn{\$docker run <image>}{запустить образ, будет загружен если его нет}
\ccn{\$docker run hello-world}{запустить базовый контейнер hello-world}
\ccn{\$docker run <image> <command>}{запустить образ и в нем команду}

\noindent Дополнительные параметры: \\
\begin{tabular}{l l}
\indent \texttt{-d} & запустить в режиме ``detach mode'' \\
\indent \texttt{-it} & запустить интерактивно (i - interactive, t - terminal) \\
\indent \texttt{{-}-rm} & удалить контейнер после завершения \\
\indent \texttt{{-}-name <name>} & задать имя для нового контейнера \\
\end{tabular}

\subsection{Контейнеры}

\ccn{\$docker ps}{список запущенных контейнеров}
\ccn{\$docker ps -a}{список всех контейнеров}
\ccn{\$docker rm <container>}{удалить контейнер}
\ccn{\$docker container prune}{удалить все контейнеры}
\ccn{\$docker start <container>}{запустить контейнер}

\tc{\$docker exec -it <container> <command>}{Выполнить интерактивно команду в контейнере:}

\ccn{\$docker pause container}{поставить на паузу}
\ccn{\$docker unpause container}{возобновить работу}
\ccn{\$docker stop container}{остановить работу}
\ccn{\$docker kill container}{убить}

\ccn{\$docker logs <container>}{Последние log'и контейнера}
\ccn{\$docker logs -f <container>}{life-logs}

\ccn{\$docker inspect <container>}{просмотр подробной информации}
\ccn{\$docker container inspect <container>}{или так}
\ccn{\$docker stats <container>}{статистика использования ресурсов}


\tc{\$docker run -d {-}-rm {-}-name MyContainer ubuntu:20.04 echo ``Hello''}{Пример команды:}
