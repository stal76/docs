
\subsection{Просмотр IP адресов}

\cn{\$ ip a}
\cn{\$ ip addr show}
\ccn{\$ ip -br a show}{в кратком виде}
\ccn{\$ ip a show enp0s3}{по одному интерфейсу}
\ccn{\$ ip a show dev enp0s3}{по одному интерфейсу}
\ccn{\$ ip a show dev enp0s3 permanent}{только статические}
\ccn{\$ ip a show dev enp0s3 dynamic}{только динамические}

\subsection{Добавление IP адреса}

\cn{\# ip addr add <ip>/<mask> dev <device>}
\cn{\# ip addr add 10.0.2.100/255.255.255.0 dev enp0s3}
\cn{\# ip addr add 10.0.2.100/24 dev enp0s3}

\subsection{Удаление IP адреса}
\cn{\# ip addr del 10.0.2.100/255.255.255.0 dev enp0s3}
\ccn{\# ip a flush}{удалить все адреса}
\ccn{\# ip a flush to 10.0.2.0/24}{удалить адреса подсети}

\subsection{Управление интерфейсами}
\cn{\$ ip l}
\cn{\$ ip link show}
\tc{\# ip link set dev <device> <action>}{Включение или выключение интерфейсов:}
\ci{\# ip link set dev enp0s3 down}
\ci{\# ip link set dev enp0s3 up}
\tc{\# ip link set mtu 4000 dev enp0s3}{Настройка MTU:}
\tc{\# ip link set dev enp0s3 address AA:BB:CC:DD:EE:FF}{Настройка MAC адреса:}
\noindent Bridges: \\
\cci{\# ip link add <name> type bridge}{добавить brdige}
\cci{\# ip link delete <name>}{удалить интерфейс}
\cci{\# ip link set <name> master <bridge>}{добавить link в bridge}

\subsection{Таблица ARP}
\cn{\$ ip neigh show}
\cn{\$ ip n}
\tc{\# ip neigh add 192.168.0.105 lladdr b0:be:76:43:21:41 dev enp0s3}{Добавление записи в таблицу ARP:}
\tc{\# ip neigh del dev enp0s3 192.168.0.105}{Очистка таблицы ARP:}
\tc{\# ip neigh flush dev enp0s3}{Удалить все записи для определённого сетевого интерфейса:}
\tc{\# ip neigh flush}{Очистить таблицу полностью:}

\subsection{Таблица маршрутизации}
\cn{\$ ip route show}
\cn{\$ ip r}
\tc{\# ip route add подсеть/маска via шлюз}{Добавление маршрута:}
\ci{\# ip route add подсеть/маска dev устройство}
\ci{\# ip route add 169.255.0.0/16 via 169.254.19.153}
\ci{\# ip route add 169.255.0.0/16 dev enp0s3}
\cci{\# ip route add default via 10.100.0.1}{по умолчанию}
\tc{\# ip route del 169.255.0.0/16 via 169.254.19.153}{Удаление маршрута:}

\subsection{Namespaces}
\ccn{\# ip netns add <name>}{добавить новый namespace}
\ccn{\# ip netns list}{список namespaces}
\ccn{\# ip -n <name> a}{список сетевых устройств в namespace}
\tc{\# ip -n <name> addr add 127.0.0.1/8 dev lo}{Задать адрес сетевому устройству в namespace:}
\tc{\# ip netns exec <name> python3 -m http.server}{Запускаем http-сервер в рамках нового netns:}

