
\section{Введение}

\subsection{Настройки}

Сохранить локальные настройки: \\
\indent \texttt{\$git config user.name <user\_name>} \\
\indent \texttt{\$git config user.email <user\_email>} \\

\noindent Варианты настроек: \\
\indent \texttt{\$git config {-}-system \ldots \indent \#} Системные: /etc/git/config \\
\indent \texttt{\$git config {-}-global \ldots \indent \#} Пользователя: ~/.gitconfig \\
\indent \texttt{\$git config {-}-local \ldots \indent \#}  Проекта (default):	<project>/.git/config \\

\noindent\texttt{\$git config {-}-list \indent \#} - все конфигурационные параметры \\
\texttt{\$git config {-}-unset user.name \indent \#} - сбросить параметр \\ 
\texttt{\$git config {-}-remove-section user \indent \#} - удалить секцию \\ 
\texttt{\$git config {-}-global core.editor \ldots \indent \#} редактор для commit \\

\noindent Алиасы, пример: \\
\indent\texttt{\$git config {-}-global alias.c 'config {-}-global'} \\
\indent\texttt{\$git c {-}-list}


\subsection{Простые команды}

\texttt{\$git init \indent \#} - инициализация каталога проекта \\
\texttt{\$git status \indent \#} - текущий статус \\
\texttt{\$git add <file> \indent \#} - добавить файл в индекс \\
Сделать commit:\\
\indent\texttt{\$git commit \indent \#} - запустится редактор для ввода комментария \\
\indent\texttt{\$git commit -m <comment>} \\
\texttt{\$git add {-}- chmod=+x <file> \indent \#} - сделать файл исполняемым \\

\noindent Специальные файлы:\\
\indent.gitignore \indent - содержит список игнорируемых файлов \\
\indent.gitkeep \indent - в пустых каталогах, тогда они добавляются в индекс\\

\noindent\texttt{\$git reset HEAD <file> \indent \#} - убрать изменения в файле \\
\texttt{\$git  add -f <file> \indent \#} - добавить файл из игнорируемого каталога\\


\texttt{\$git add -p <file> \indent \#} - с вопросом по каждому изменению \\

\noindent Добавление всех изменений сразу в commit:\\
\indent \texttt{\$git commit -all \indent} \\
\indent \texttt{\$git commit -am <comment>} \\
\indent \texttt{\$git commit -m <comment> file1 file2 \ldots" \#} - несколько файлов \\

\noindent Удалить файл: \\
\indent \texttt{\$git rm <file> \indent \#} - удаление из каталога и индекса \\
\indent \texttt{\$git rm -r <file> \indent \#} - удаление из каталога\\
\indent \texttt{\$git rm {-}-cached <file> \indent \#} - удаление из индекса \\
\indent \texttt{\$git rm -f <file> \indent \#} - удаление с несохраненными изменениями\\

\noindent Переименовать файл:\\
\indent \texttt{\$git mv <old\_name> <new\_name>}