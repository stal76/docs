\section{Reset}

\noindent Сброс (по умолчанию mix): \\
\indent \texttt{\$git reset {-}-hard <commit> \indent\#} - Жесткий сброс \\
\indent \texttt{\$git reset {-}-mix <commit> \indent\#} - Смешанный сброс \\
\indent \texttt{\$git reset {-}-soft <commit> \indent\#} - Мягкий сброс \\
\indent \indent 	\texttt{@ \indent} - последний commit \\
\indent \indent 	\texttt{@\~ \indent} - предыдущий commit \\

Отличия видов reset (из \texttt{\$git reset {-}{-}help}): \\

\texttt {
	\begin{tabular}{c c c c l c c c}
		working & index & HEAD & target &  & working & index & HEAD \\
		\hline
		A & B & C & D & {-}{-}soft & A & B & D \\
		&  & & & {-}{-}mixed & A & D & D \\
		&  & & & {-}{-}hard & D & D & D \\
	\end{tabular}
}


\noindent \texttt{\$git reflog master} \\

\noindent Взять комментарий из другого commit: \\
\indent \texttt{\$git commit -c <commit> \indent\#} - откроется редактор  \\
\indent \texttt{\$git commit -C <commit> \indent\#} - не будет открываться редактор \\
\indent \indent \texttt{ {-}-reset-author \indent} - сбросить автора (по умолчанию из commit) \\

\noindent \texttt{\$git commit {-}-amend \indent\#}  - исправить в последнем commit \\
\indent \texttt{ {-}-no-edit\indent\#} - не открывать редактор \\ 
\indent \texttt{ {-}-reset-author} \\

\noindent Очистка индекса от всех изменений: \\
\indent \texttt{\$git reset HEAD} \\
\noindent Удалить изменения по файлу в индексе \\
\indent \texttt{\$git reset <file>} \\
\noindent Поместить файл из commit в индекс (в директории не меняется):\\
\indent \texttt{\$git reset <commit> <file>} \\
