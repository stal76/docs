\section{Просмотр истории}

\noindent \texttt{\$git log \indent\#} - Вывод истории commit достижимых из HEAD \\
\noindent \texttt{\$git log {-}-name-only\indent\#} - Выводить имена файлов \\
\noindent \texttt{\$git log -n <count>\indent\#} - Вывести count последних commit \\
\noindent \texttt{\$git log {-}-pretty=<value>} \\
\indent \texttt{medium (default)} \\
\indent\texttt{oneline} \\
\indent\texttt{format:'\%h \%cd | \%s\%d [\%an]' }  - id, date | comment ссылки автор \\
\indent\indent\texttt{\%C(<colorname)\indent} - задать цвет \\


\indent \texttt{{-}-abbrev-commit \indent\#} - обрезать id commit \\

\noindent Пример вывода лога в определенном формате: \\
\indent \texttt{\$git log {-}-reverse {-}-pretty=format:'\%Cred\%h\%Creset -\%C(yellow)\%d\%Creset \%s \%Cgreen(\%cr) \%C(bold blue)<\%an>\%Creset' {-}-abbrev-commit} \\


\noindent Можно задать в конфигурации: \\
\indent \texttt{\$git config format.pretty '\%Cred\%h\%Creset -\%C(yellow)\%d\%Creset \%s \%Cgreen(\%cr) \%C(bold blue)<\%an>\%Creset'} \\


\noindent \texttt{\$git log {-}-oneline}  = \texttt{\$git log {-}-pretty=oneline {-}-abbrev-commit} \\
\indent \texttt{{-}-decorate-short} \\
\indent \texttt{{-}-date=short \indent\#} - только даты \\
\indent \texttt{{-}-date=format:'\%F \%R' \indent\#}  - берет флаги из strftime \\

\noindent \texttt{\$git log <branch> \indent\#}  - log по commit достижимым из ветки \\
\indent \texttt{{-}-graph \indent\#} - в виде дерева \\
\indent \texttt{{-}-all \indent\#} - из всех commit \\

\noindent \texttt{\$git log gui} \\
\noindent \texttt{\$git log feature {$ \mathbin{\char`\^} $}master \indent\#} - log по feature кроме master \\
\noindent \texttt{\$git log master..feature} \\
\indent \texttt{{-}-boundary \indent\#} - включая пограничный commit \\


\noindent \texttt{\$git log ..feature} - симметрическая разница \\

\noindent \texttt{\$git log <file>} - история по файлу \\
\indent \texttt{-p} - показывать различия \\
\indent \texttt{{-}-follow} - если был переименован, то и по старым названиям \\

\noindent История между commit по списку файлов: \\
\indent \texttt{\$git log <commit1>..<commit2> <file1> <file2> ...} \\

\noindent Поиск по строке по commit достижимым из HEAD: \\
\indent \texttt{\$git log {-}-grep <string>}\\
\noindent \texttt{\$git log {-}-grep <str1> {-}-grep <str2> \#} - str1 или str2 в сообщении \\
\indent \indent \texttt{{-}-all-match \indent} - по всем совпадение \\
\indent \indent \texttt{-p \indent} - Perl type совместимые регулярные выражения \\
\indent \indent \texttt{-F \indent} - отключить регулярные выражения \\
\indent \indent \texttt{-i \indent} - регистро-независимый поиск \\

\noindent Поиск по содержимому: \\
\indent \texttt{\$git log -G <string>}\\
\noindent Все изменения со строки i1 по i2 в <file>: \\
\indent \texttt{\$git log -L i1,i2:<file>}\\
\noindent Регулярные выражения для начала и конца фрагмента: \\
\indent \texttt{\$git log -L r1,r2:<file> - r1,r2}\\
\noindent Регулярное выражение <func> для имени функции: \\
\indent \texttt{\$git log :<func>:<file>} \\

\noindent \texttt{\$git log}: \\
\indent \texttt{{-}-author} \\
\indent \texttt{{-}-commiter} \\
\indent \texttt{{-}-before 'before 3 months'} \\
\indent \texttt{{-}-after} \\

\noindent \texttt{\$git blame <file> \indent\#} - кто менял строки файла \\