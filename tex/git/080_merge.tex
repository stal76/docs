\section{Слияние}

\noindent \texttt{\$git merge <branch> \indent\#} - объединить ветки \\

\noindent При конфликте взять версию файла: \\
\indent \texttt{\$git checkout {-}-ours <file> \indent\#} - из текущей ветки  \\
\indent \texttt{\$git checkout {-}-theirs <file> \indent\#} - из объединяемой ветки \\

\noindent \texttt{\$git checkout {-}-merge} \\

\noindent Вернуть в состоянии до merge: \\
\indent \texttt{\$git reset {-}-hard} \\
\noindent Оставить которые хорошо объединились, проблемные вернуть в начальное состояние: \\
\indent \texttt{\$git reset {-}-merge} \\

\noindent \texttt{\$git checkout {-}-conflict-diff3 {-}-merge <file>} \\

\noindent После окончания разрешения конфликтов создает окончательный commit слияния: \\
\indent \texttt{\$git merge {-}-continue}

\noindent id сливаемого commit в .git/MERGE\_HEAD \\

\noindent Отмена слияния: \\
\indent \texttt{\$git reset {-}-hard {@}\~} \\

\noindent Слияние без автоматического commit: \\
\indent \texttt{\$git merge <branch> {-}-no-commit} \\

\noindent Слияние без Fast Forward (перемотки), создается новый commit: \\
\indent \texttt{\$git merge {-}-no-ff {-}-no-edit <commit>} \\

\noindent Слияние без сохранения истории сливаемого commit: \\
\indent \texttt{\$git merge {-}-squash <commit>} \\