\subsection{Модули ядра Linux}

По материалам \href{https://losst.pro/moduli-yadra-linux}{отсюда} \\
\tc{/lib/modules/5.4.0.45-generic/kernel/}{Модули ядра расположены:}

\noindent Основные команды: \\
\indent\texttt{lsmod} - посмотреть загруженные модули \\
\indent\texttt{modinfo} - информация о модуле \\
\indent\texttt{insmod} - загрузить модуль \\
\indent\texttt{rmmod} - удалить модуль \\

\noindent Все установленные модули ядра: \\
\indent\texttt{dpkg -S *.ko | grep /lib/modules} \\
\indent\texttt{find /lib/modules -name *.ko} \\
\indent\texttt{modprobe -c} \\

\tc{find /lib/modules/\$(uname -r) -name *.ko}{Для текущей версии ядра:}

\noindent Загруженные модули: \\
\indent\texttt{cat /proc/modules} \\
\indent\texttt{sudo lsmod} \\


\tc{modinfo fuse}{Подробная информация о модуле:}

\noindent Запуск модулей ядра: \\
\indent\texttt{sudo modprobe vboxdrv} \\
\indent\texttt{sudo insmod /lib/modules/4.1.20-11-default/weak-updates/misc/vboxdrv.ko} \\

\noindent Удаление модулей ядра: \\
\indent\texttt{sudo modprobe -r vboxdrv} \\
\indent\texttt{sudo rmmod vboxdrv} \\

\tc{sudo vi /etc/modprobe.d/blacklist.conf}{Блокирование загрузки модулей:}

\tc{sudo vi /etc/modules.load.d/modules.conf}{Автозагрузка модулей:}

