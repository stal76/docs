\subsection{awk}

\tc{\$awk program input-files}{Выполнить программу:}
\tc{\$awk -f program-file1 -f program-file2 -f program-file3 input-files}{Выполнить несколько программ:}

\noindent Слово \texttt{BEGIN} - действие перед обработкой ввода команды \texttt{awk}. \\
\noindent Слово \texttt{END} - действие после обработки ввода команды \texttt{awk}. \\

\ccn{\$awk 'FNR<=10' myfile}{Выводит 10 строк и завершается}

\tc{\$echo "linux efficient"\ | awk '\{print \$2, \$1\}'}{Поменять местами два слова:}
\tc{\$awk '\{print \$2\}' /etc/hosts}{Печать второго слова в каждой строке:}

\ccn{\$echo Efficient fun Linux | awk '\{print \$1 \$3\}'}{без пробела}
\ccn{\$echo Efficient fun Linux | awk '\{print \$1, \$3\}'}{с пробелом}

\tc{\$df / /data | awk ' FNR>1 \{print \$4\}'}{Выводить со второй строки 4 колонку:}
\tc{\$echo efficient:::::linux | awk -F':*' '\{print \$2\}'}{Любое количество двоеточий:}

\ccn{\$awk '\{print \$NF\}' celebrities}{вывести последнее слово}
\ccn{\$echo efficient linux | awk '/efficient/'}{вхождения строки}
\ccn{awk: \$3\textasciitilde/\^{}[A-Z]/}{начинается ли третье поле с заглавной буквы}

\noindent Пример программы:\\
\ci{\$awk -F'$\backslash$t' $\backslash$}
\indent\ci{' BEGIN \{print "Recent books:"\} $\backslash$}
\indent\ci{\$3\textasciitilde/\^{}201/\{print "{}-"{},\ \$4, "("\ \$3 ")."{}, "$\backslash$"{}"\ \$2 "$\backslash$"{}"\} $\backslash$}
\indent\ci{END \{print "For more books, search the web"\} ' $\backslash$}
\indent\ci{animals.txt}

\tc{\$seq 1 100 | awk '\{s+=\$1\} END \{print s\}'}{Просуммировать числа от 1 до 100:}

\noindent Найти дубликать картинок: \\
\ci{\$ md5sum *.jpg $\backslash$ }
\ci{| awk '{counts[\$1]++; names[\$1]=names[\$1] "{}\ "{}\ \$2} $\backslash$ }
\ci{END \{for (key in counts) print counts[key] "{}\ "{} key "{}:"{} names[key]\}'$\backslash$}
\ci{| grep -v '\^{}1 ' $\backslash$}
\ci{| sort -nr}