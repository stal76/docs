\documentclass[12pt, a4paper]{article}
\usepackage[utf8]{inputenc}
\usepackage[russian]{babel}
\usepackage{cmap}
\usepackage{amsmath}
\usepackage{amsfonts}
\usepackage{amssymb}
\usepackage{makeidx}
\usepackage{graphicx}
\usepackage{hyperref}
\usepackage{makecell}
\usepackage[table]{xcolor}

% Tables in latex
% https://latex-tutorial.com/tables-in-latex/

\newcommand{\ci}[1]{\indent\texttt{#1} \\}
\newcommand{\cn}[1]{\noindent\texttt{#1} \\}
\newcommand{\cci}[2]{\indent\texttt{#1\indent \#} - #2 \\}
\newcommand{\ccn}[2]{\noindent\texttt{#1\indent \#} - #2 \\}
\newcommand{\tc}[2]{\noindent #2 \\ \indent\texttt{#1} \\}

\begin{document}
	
	\begin{center} {\Huge Linux command line} \end{center}

	
	\section{cut}

\ccn{\$cut -f2 animals.txt}{вырезать поле в каждой строке}
\ccn{\$cut -f1,3 animals.txt}{вырезать несколько полей}
\ccn{\$cut -f2-4 animals.txt}{указав диапазон}

\tc{\$cut -d: -f1 /etc/passwd | sort}{Вывести все имена пользователей и отсортировать их:}


\ccn{\$cut -c1-3 animals.txt}{по положению символа в строке}
	\section{date}

\ccn{\$date +\%Y-\%m-\%d}{Формат год-месяц-день: 2021-06-28}
\ccn{\$date +\%H:\%M:\%S}{Формат часы:минуты:секунды 16:57:33}
\ccn{\$date +"it's already \%A!"}{it's already Tuesday!}
	\section{seq}

\ccn{\$seq 1 5}{Выводит все целые числа от 1 до 5 включительно}
\ccn{\$seq 1 2 10}{Увеличение на 2 вместо 1}
\ccn{\$seq 3 -1 0}{отрицательный шаг}
\ccn{\$seq 1.1 0.1 2}{Увеличение на 0,1}
\ccn{\$seq -s/ 1 5}{Разделение значений с помощью косой черты}
\ccn{\$seq -w 8 10}{приводит все значения к одинаковой ширине}


	\section{Расширение команд с помощью фигурных скобок}

\ccn{\$echo \{1..10\}}{Вперед, начиная с 1: 1 2 3 4 5 6 7 8 9 10}
\ccn{\$echo \{10..1\}}{Назад, начиная с 10}
\ccn{\$echo \{01..10\}}{С ведущими нулями (для равной ширины)}
\ccn{\$echo \{1..1000..100\}}{Приращение сотнями, начиная с 1}
\ccn{\$echo \{1000..1..100\}}{Уменьшение сотнями, начиная с 1000}
\ccn{\$echo \{01..1000..100\}}{С ведущими нулями}

\noindent Фигурные скобки vs квадратные: \\
\cci{\$ls file[2-4]}{Соответствует существующим именам файлов}
\cci{\$ls file\{2..4\}}{Вычисляется в: file2 file3 file4}

\ccn{\$echo \{A..Z\}}{A B C \ldots X Y Z}
\ccn{\$echo \{A..Z\} | tr -d ' '}{Удалить пробелы: ABC\ldots XYZ}
	\section{find}

\ccn{\$find /etc -print}{Список всех каталогов в /etc рекурсивно}
\ccn{\$find . -type f -print}{Только файлы}
\ccn{\$find . -type d -print}{Только каталоги}
\tc{\$find /etc -type f -name "*.conf"\ -print}{Файлы, заканчивающиеся на .conf:}
\tc{\$find . -iname "*.txt"\ -print}{Шаблон нечувствительный к регистру:}

\noindent Выполнить команду для каждого найденного файла, в конце обязательно ";"\ или экранированную ${\backslash}$; \\
\ci{\$find /etc -exec echo @ \{\} @ ";"  }
\ci{\$find /etc -type f -name "*.conf"\ -exec ls -l \{\} "\;"}
	\section{yes}

\ccn{\$yes}{Выводит «y» по умолчанию}
\ccn{\$yes woof!}{Повторять любую другую строку}
	\section{grep}

\ccn{\$grep his frost}{Вывести строки, содержащие «his»}
\ccn{\$grep -w his frost}{Искать точное соответствие «his»}
\ccn{\$grep -i his frost}{игнорировать регистр букв}
\ccn{\$grep -l his *}{В каком файле содержится «his»?}

\begin{tabular}{|p{5cm}|c|p{6cm}|}
\hline
\rowcolor{gray!40}
\centering Соответствие & \makecell{Используемые \\ выражения} & \makecell{Пример} \\
\hline
Начало строки & \texttt{\^{}} & 
\texttt{\^{}a} = строка, начинающаяся с a \\
\hline
Конец строки & \texttt{\$} & 
\makecell{\texttt{!\$} = строка, заканчивающаяся \\ восклицательным знаком} \\
\hline
Любой одиночный символ (кроме новой строки) & \texttt{.} &
\texttt{...} = любые три последовательных символа \\
\hline
Знаки вставки, доллара или любой другой специальный символ с &
\texttt{${\backslash}$c} &
\texttt{\$} = знак доллара \\
\hline
Ноль или более вхождений выражения E & \texttt{E*} &
\texttt{\_*} = ноль или более знаков подчеркивания \\
\hline
Любой одиночный символ в наборе & \texttt{[characters]} &
\texttt{[aeiouAEIOU]} = любая гласная \\
\hline

Любой одиночный символ, не входящий в набор &
\texttt{[\^{}characters]}  &
\texttt{[\^{}aeiouAEIOU]} = любая не гласная \\

\hline

Любой символ в диапазоне  между с1 и с2 &
\texttt{[c1-c2]} &
\texttt{[0-9]} = любая цифра \\

\hline

Любой символ вне диапазона между c1 и c2 &
\texttt{[\^{}c1-c2]} &
\texttt{[\^{}0-9]} = любой нецифровой символ \\

\hline

Любое из двух выражений E1 или E2 &
\makecell{\texttt{E1$\backslash$|E2} \\ для grep и sed, \\ \texttt{E1/E2} для awk} &
\texttt{one|two} = или one, или two \\

\hline

Группировка выражения E с учетом приоритета &
\makecell{\texttt{$\backslash$(E$\backslash$)} \\ для grep и sed,\\ \texttt{(E)} для awk} & 
\makecell[l]{\texttt{$\backslash$(one$\backslash$|two)$\backslash$*} \\ \texttt{(one|two)*} = ноль или более \\ вхождений one или two} \\

\hline

\end{tabular}
\\ 

\ccn{\$grep -v '\^{}\$' myfile}{все непустые строки, \texttt{-v} - исключает}
\ccn{\$grep 'cookie$\backslash$|cake' myfile}{содержащие либо cookie, либо cake}
\ccn{\$grep '<.*>' page.html}{< появляется перед символом >}

\ccn{\$grep -F w. frost}{отключить регулярные выражения}
\ccn{\$fgrep w. frost}{поиск без регулярных выражений}

\tc{\$grep -f <filename> ...}{Поиск по списку шаблонов из файла:}
	\section{awk \{print\}}

\tc{\$awk '\{print \$2\}' /etc/hosts}{Печать второго слова в каждой строке:}

\ccn{\$echo Efficient fun Linux | awk '\{print \$1 \$3\}'}{без пробела}
\ccn{\$echo Efficient fun Linux | awk '\{print \$1, \$3\}'}{с пробелом}

\tc{\$df / /data | awk ' FNR>1 \{print \$4\}'}{Выводить со второй строки:}

\tc{\$echo efficient:::::linux | awk -F':*' '\{print \$2\}'}{Любое количество двоеточий:}

	\section{head, tail, tac}

\ccn{\$head -n3 animals.txt}{первые три строки файла}

\ccn{\$tail -n3 alphabet}{последние 3 строки}
\ccn{\$tail -n+25 alphabet}{с 25-й строки файла}
\ccn{\$head -n4 alphabet | tail -n1}{только четвертую строку}
\ccn{\$head -n8 alphabet | tail -n3}{строки с шестой по восьмую}

\ccn{\$head -4 alphabet}{= \texttt{head -n4 alphabet}}
\ccn{\$tail -3 alphabet}{= \texttt{tail -n3 alphabet}}
\ccn{\$tail +25 alphabet}{= \texttt{tail -n+25 alphabet}}

\ccn{\$tac filename}{вывести строки в обратном порядке}
	\section{paste}

\tc{\$paste title-words1 title-words2}{Объединить строки в столбцы, разделенные символом табуляции:}

\ccn{\$paste -d, title-words1 title-words2}{разделитель запятая}
\ccn{\$paste -d "$\backslash$n"\ title-words1 title-words2}{чередовать строки}

\tc{\$paste -d, -s title-words1 title-words2}{Строки каждого файла соединяются в одну:}

	\section{tr}

\tc{\$echo \$PATH | tr : "$\backslash$n"}{Преобразование двоеточий в символы новой строки:}
\tc{\$echo efficient | tr a-z A-Z}{Перевод a в A, b в B и т. д.}
\tc{\$ echo Efficient Linux | tr "\ "\ "$\backslash$n"}{Преобразование пробелов в символы новой строки:}
\tc{\$ echo efficient linux | tr -d ' $\backslash$t'}{Удаление пробелов и знаков табуляции}
	\subsection{rev}

\tc{\$echo Efficient Linux! | rev}{Переворачивает символы задом наперед в каждой строке ввода:}
\tc{\$rev celebrities | cut -d' ' -f1 | rev}{Вывести на экран последнее слово из каждой строки:}


\section{Неразобранное}
\noindent Полезные команды: \\
\cci{\$cat /etc/os-release}{информация о версии системы}
\cci{\$netstat -tulpen}{Открытые порты}
\cci{\$curl -li https://localhost}{Содержимое сайта}
\cci{env}{получить все системные переменные окружения}
\cci{export <var>=<value>}{задать переменнуж окружения}
	
\end{document}
