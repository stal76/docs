\documentclass[12pt, a4paper]{article}
\usepackage[utf8]{inputenc}
\usepackage[russian]{babel}
\usepackage{cmap}
\usepackage{amsmath}
\usepackage{amsfonts}
\usepackage{amssymb}
\usepackage{makeidx}
\usepackage{graphicx}
\usepackage{hyperref}
\usepackage{makecell}
\usepackage[table]{xcolor}
\usepackage{verbatim}

% Tables in latex
% https://latex-tutorial.com/tables-in-latex/

\newcommand{\ci}[1]{\indent\texttt{#1} \\}
\newcommand{\cn}[1]{\noindent\texttt{#1} \\}
\newcommand{\cci}[2]{\indent\texttt{#1\indent \#} - #2 \\}
\newcommand{\ccn}[2]{\noindent\texttt{#1\indent \#} - #2 \\}
\newcommand{\tc}[2]{\noindent #2 \\ \indent\texttt{#1} \\}

\begin{document}
	
	\begin{center} {\Huge Linux command line} \end{center}
	
	\tableofcontents
	
	\section{date}

\ccn{\$date +\%Y-\%m-\%d}{Формат год-месяц-день: 2021-06-28}
\ccn{\$date +\%H:\%M:\%S}{Формат часы:минуты:секунды 16:57:33}
\ccn{\$date +"it's already \%A!"}{it's already Tuesday!}
	\section{seq}

\ccn{\$seq 1 5}{Выводит все целые числа от 1 до 5 включительно}
\ccn{\$seq 1 2 10}{Увеличение на 2 вместо 1}
\ccn{\$seq 3 -1 0}{отрицательный шаг}
\ccn{\$seq 1.1 0.1 2}{Увеличение на 0,1}
\ccn{\$seq -s/ 1 5}{Разделение значений с помощью косой черты}
\ccn{\$seq -w 8 10}{приводит все значения к одинаковой ширине}


	\section{Расширение команд с помощью фигурных скобок}

\ccn{\$echo \{1..10\}}{Вперед, начиная с 1: 1 2 3 4 5 6 7 8 9 10}
\ccn{\$echo \{10..1\}}{Назад, начиная с 10}
\ccn{\$echo \{01..10\}}{С ведущими нулями (для равной ширины)}
\ccn{\$echo \{1..1000..100\}}{Приращение сотнями, начиная с 1}
\ccn{\$echo \{1000..1..100\}}{Уменьшение сотнями, начиная с 1000}
\ccn{\$echo \{01..1000..100\}}{С ведущими нулями}

\noindent Фигурные скобки vs квадратные: \\
\cci{\$ls file[2-4]}{Соответствует существующим именам файлов}
\cci{\$ls file\{2..4\}}{Вычисляется в: file2 file3 file4}

\ccn{\$echo \{A..Z\}}{A B C \ldots X Y Z}
\ccn{\$echo \{A..Z\} | tr -d ' '}{Удалить пробелы: ABC\ldots XYZ}
	\section{find}

\ccn{\$find /etc -print}{Список всех каталогов в /etc рекурсивно}
\ccn{\$find . -type f -print}{Только файлы}
\ccn{\$find . -type d -print}{Только каталоги}
\tc{\$find /etc -type f -name "*.conf"\ -print}{Файлы, заканчивающиеся на .conf:}
\tc{\$find . -iname "*.txt"\ -print}{Шаблон нечувствительный к регистру:}

\noindent Выполнить команду для каждого найденного файла, в конце обязательно ";"\ или экранированную ${\backslash}$; \\
\ci{\$find /etc -exec echo @ \{\} @ ";"  }
\ci{\$find /etc -type f -name "*.conf"\ -exec ls -l \{\} "\;"}
	\section{yes}

\ccn{\$yes}{Выводит «y» по умолчанию}
\ccn{\$yes woof!}{Повторять любую другую строку}
	\section{grep}

\ccn{\$grep his frost}{Вывести строки, содержащие «his»}
\ccn{\$grep -w his frost}{Искать точное соответствие «his»}
\ccn{\$grep -i his frost}{игнорировать регистр букв}
\ccn{\$grep -l his *}{В каком файле содержится «his»?}

\begin{tabular}{|p{5cm}|c|p{6cm}|}
\hline
\rowcolor{gray!40}
\centering Соответствие & \makecell{Используемые \\ выражения} & \makecell{Пример} \\
\hline
Начало строки & \texttt{\^{}} & 
\texttt{\^{}a} = строка, начинающаяся с a \\
\hline
Конец строки & \texttt{\$} & 
\makecell{\texttt{!\$} = строка, заканчивающаяся \\ восклицательным знаком} \\
\hline
Любой одиночный символ (кроме новой строки) & \texttt{.} &
\texttt{...} = любые три последовательных символа \\
\hline
Знаки вставки, доллара или любой другой специальный символ с &
\texttt{${\backslash}$c} &
\texttt{\$} = знак доллара \\
\hline
Ноль или более вхождений выражения E & \texttt{E*} &
\texttt{\_*} = ноль или более знаков подчеркивания \\
\hline
Любой одиночный символ в наборе & \texttt{[characters]} &
\texttt{[aeiouAEIOU]} = любая гласная \\
\hline

Любой одиночный символ, не входящий в набор &
\texttt{[\^{}characters]}  &
\texttt{[\^{}aeiouAEIOU]} = любая не гласная \\

\hline

Любой символ в диапазоне  между с1 и с2 &
\texttt{[c1-c2]} &
\texttt{[0-9]} = любая цифра \\

\hline

Любой символ вне диапазона между c1 и c2 &
\texttt{[\^{}c1-c2]} &
\texttt{[\^{}0-9]} = любой нецифровой символ \\

\hline

Любое из двух выражений E1 или E2 &
\makecell{\texttt{E1$\backslash$|E2} \\ для grep и sed, \\ \texttt{E1/E2} для awk} &
\texttt{one|two} = или one, или two \\

\hline

Группировка выражения E с учетом приоритета &
\makecell{\texttt{$\backslash$(E$\backslash$)} \\ для grep и sed,\\ \texttt{(E)} для awk} & 
\makecell[l]{\texttt{$\backslash$(one$\backslash$|two)$\backslash$*} \\ \texttt{(one|two)*} = ноль или более \\ вхождений one или two} \\

\hline

\end{tabular}
\\ 

\ccn{\$grep -v '\^{}\$' myfile}{все непустые строки, \texttt{-v} - исключает}
\ccn{\$grep 'cookie$\backslash$|cake' myfile}{содержащие либо cookie, либо cake}
\ccn{\$grep '<.*>' page.html}{< появляется перед символом >}

\ccn{\$grep -F w. frost}{отключить регулярные выражения}
\ccn{\$fgrep w. frost}{поиск без регулярных выражений}

\tc{\$grep -f <filename> ...}{Поиск по списку шаблонов из файла:}
	\section{less}

Сочетания клавиш при просмотре с помощью команды less:

\begin{tabular}{|c|l|}
\hline
\rowcolor{gray!40}
Сочетание & Действие \\
\hline
h & Справка \\
Пробел & Посмотреть следующую страницу \\
b & Посмотреть предыдущую страницу \\
Enter & Прокрутить вниз на одну строку \\
< & Перейти к началу документа \\
> & Перейти к концу документа \\
/ & Поиск текста вперед (введите текст и нажмите Enter) \\
? & Поиск текста назад (введите текст и нажмите Enter) \\
n & Найти следующее вхождение искомого текста \\
q & Выйти \\
\hline
\end{tabular}

	\section{Вывод текста}
	\section{head, tail, tac}

\ccn{\$head -n3 animals.txt}{первые три строки файла}

\ccn{\$tail -n3 alphabet}{последние 3 строки}
\ccn{\$tail -n+25 alphabet}{с 25-й строки файла}
\ccn{\$head -n4 alphabet | tail -n1}{только четвертую строку}
\ccn{\$head -n8 alphabet | tail -n3}{строки с шестой по восьмую}

\ccn{\$head -4 alphabet}{= \texttt{head -n4 alphabet}}
\ccn{\$tail -3 alphabet}{= \texttt{tail -n3 alphabet}}
\ccn{\$tail +25 alphabet}{= \texttt{tail -n+25 alphabet}}

\ccn{\$tac filename}{вывести строки в обратном порядке}
	\subsection{cut}

\ccn{\$cut -f2 animals.txt}{вырезать поле в каждой строке}
\ccn{\$cut -f1,3 animals.txt}{вырезать несколько полей}
\ccn{\$cut -f2-4 animals.txt}{указав диапазон}

\tc{\$cut -d: -f1 /etc/passwd | sort}{Вывести все имена пользователей и отсортировать их:}


\ccn{\$cut -c1-3 animals.txt}{по положению символа в строке}
	
	\section{Преобразование текста}
	\subsection{tr}

\tc{\$echo \$PATH | tr : "$\backslash$n"}{Преобразование двоеточий в символы новой строки:}
\tc{\$echo efficient | tr a-z A-Z}{Перевод a в A, b в B и т. д.}
\tc{\$ echo Efficient Linux | tr "\ "\ "$\backslash$n"}{Преобразование пробелов в символы новой строки:}
\tc{\$ echo efficient linux | tr -d ' $\backslash$t'}{Удаление пробелов и знаков табуляции}
	\section{rev}

\tc{\$echo Efficient Linux! | rev}{Переворачивает символы задом наперед в каждой строке ввода:}
\tc{\$rev celebrities | cut -d' ' -f1 | rev}{Вывести на экран последнее слово из каждой строки:}
	
	\section{paste}

\tc{\$paste title-words1 title-words2}{Объединить строки в столбцы, разделенные символом табуляции:}

\ccn{\$paste -d, title-words1 title-words2}{разделитель запятая}
\ccn{\$paste -d "$\backslash$n"\ title-words1 title-words2}{чередовать строки}

\tc{\$paste -d, -s title-words1 title-words2}{Строки каждого файла соединяются в одну:}

	\section{awk}

\tc{\$awk program input-files}{Выполнить программу:}
\tc{\$awk -f program-file1 -f program-file2 -f program-file3 input-files}{Выполнить несколько программ:}

\noindent Слово \texttt{BEGIN} - действие перед обработкой ввода команды \texttt{awk}. \\
\noindent Слово \texttt{END} - действие после обработки ввода команды \texttt{awk}. \\

\ccn{\$awk 'FNR<=10' myfile}{Выводит 10 строк и завершается}

\tc{\$echo "linux efficient"\ | awk '\{print \$2, \$1\}'}{Поменять местами два слова:}
\tc{\$awk '\{print \$2\}' /etc/hosts}{Печать второго слова в каждой строке:}

\ccn{\$echo Efficient fun Linux | awk '\{print \$1 \$3\}'}{без пробела}
\ccn{\$echo Efficient fun Linux | awk '\{print \$1, \$3\}'}{с пробелом}

\tc{\$df / /data | awk ' FNR>1 \{print \$4\}'}{Выводить со второй строки 4 колонку:}
\tc{\$echo efficient:::::linux | awk -F':*' '\{print \$2\}'}{Любое количество двоеточий:}

\ccn{\$awk '\{print \$NF\}' celebrities}{вывести последнее слово}
\ccn{\$echo efficient linux | awk '/efficient/'}{вхождения строки}
\ccn{awk: \$3\textasciitilde/\^{}[A-Z]/}{начинается ли третье поле с заглавной буквы}

\noindent Пример программы:\\
\ci{\$awk -F'$\backslash$t' $\backslash$}
\indent\ci{' BEGIN \{print "Recent books:"\} $\backslash$}
\indent\ci{\$3\textasciitilde/\^{}201/\{print "{}-"{},\ \$4, "("\ \$3 ")."{}, "$\backslash$"{}"\ \$2 "$\backslash$"{}"\} $\backslash$}
\indent\ci{END \{print "For more books, search the web"\} ' $\backslash$}
\indent\ci{animals.txt}

\tc{\$seq 1 100 | awk '\{s+=\$1\} END \{print s\}'}{Просуммировать числа от 1 до 100:}

\noindent Найти дубликать картинок: \\
\ci{\$ md5sum *.jpg $\backslash$ }
\ci{| awk '{counts[\$1]++; names[\$1]=names[\$1] "{}\ "{}\ \$2} $\backslash$ }
\ci{END \{for (key in counts) print counts[key] "{}\ "{} key "{}:"{} names[key]\}'$\backslash$}
\ci{| grep -v '\^{}1 ' $\backslash$}
\ci{| sort -nr}
	\subsection{sed}

\tc{\$sed script input-files}{Выполнить сценарий:}
\tc{\$sed -e script1 -e script2 -e script3 input-files}{Выполнить несколько сценариев:}
\tc{\$sed -f script-file1 -f script-file2 -f script-file3 input-files}{Выполнить несколько сценариев из файлов:}

\ccn{\$sed 10q myfile}{выводит 10 строк и завершается}
\ccn{\$echo image.jpg | sed 's/\.jpg/.png/'}{заменить .jpg на .png}
\ccn{\$sed 's/.* //' celebrities}{вывести последнее слово}

\tc{s\_one\_two\_}{Можно использовать другие символы для разделения:}
\tc{\$echo Efficient Stuff | sed "s/stuff/linux/i"}{Нечувствительный к регистру:}
\tc{\$echo efficient stuff | sed "s/f/F/g"}{Заменяет все вхождения «f»:}

\ccn{\$seq 10 14 | sed 4d}{удаляет четвертую строку}
\tc{\$seq 101 200 | sed '/[13579]\$/d'}{Удаляет строки, заканчивающиеся на нечетные цифры:}

\noindent Использование ссылок на подвыражения $\backslash$1, $\backslash$2, ... : \\
\ci{\$ls | sed "s/image$\backslash$.jpg$\backslash$.$\backslash$([1-3]$\backslash$)/image$\backslash$1.jpg/"}

	\section{11 способов запуска команды}

\subsection{Условные списки}
\ccn{\$cd dir \&\& touch new.txt}{запускаются до первой ошибки}
\ccn{\$cd dir || mkdir dir}{запускаются до первого успеха}

\subsection{Безусловные списки}
\tc{\$sleep 300; echo "remember to walk the dog"\ | mail -s reminder \$USER}{Последовательный запуск команд:}

\subsection{Подстановка команд}
\cn{\$mv \$(grep -l "Artist: Kansas" *.txt) kansas}
\cn{\$echo Today is \$(date +\%A).}
\cn{\$echo Today is \$(echo \$(date + \%A) | tr a-z A-Z )!}

\subsection{Подстановка процесса}
\cn{\$cat <(ls -1 | sort -n)}
\cn{\$cp <(ls -1 | sort -n) /tmp/listing}
\cn{\$diff <( ls *.jpg | sort -n ) <( seq 1 1000 | sed 's/\$/.jpg/' )}

\subsection{Передача команды в bash в качестве аргумента}
\cn{\$bash -c "ls -l"}
\cn{\$sudo bash -c 'echo "New log file"\ > /var/log/custom.log'}

\subsection{Передача команды в bash через стандартный ввод}
\cn{\$echo "ls -l"\ | bash}

\subsection{Удаленное выполнение однострочника с помощью ssh}
\tc{\$ssh myhost.example.com ls > outfile}{Создает outfile на локальном хосте:}
\tc{\$ssh myhost.example.com "ls > outfile"}{Создает outfile на удаленном хосте:}
\tc{\$echo "ls > outfile" | ssh -T myhost.example}{С параметром \texttt{-T}, чтобы удаленный ssh-сервер не выделял терминал:}

\subsection{Запуск списка команд с помощью xargs}
У команды find использовать \texttt{-print0} вместо \texttt{-print}, то строки будут разделяться нулевым символом. \texttt{xargs -0} - чтобы разделителем служил нулевой символ.

\cn{\$find . -type f -name $\backslash$*.py -print0 | xargs -0 wc -l}

\ccn{\$ls | xargs echo}{уместить как можно больше входных строк}
\ccn{\$ls | xargs -n1 echo}{один аргумент в каждой команде echo}
\ccn{\$ls | xargs -n3 echo}{три аргумента в каждой команде echo}

\tc{\$ls | xargs -I XYZ echo XYZ is my favorite food}{Параметр -I определяет место входных строк в сгенерированной команде. XYZ в качестве прототипа:}

\cn{\$find . -maxdepth 1 -name $\backslash$*.txt -type f -print0 $\backslash$}
\ci{| xargs -0 rm}

\subsection{Фоновое выполнение команды}
\tc{\$wc -c my\_extremely\_huge\_file.txt \&}{Подсчет символов в огромном файле:}
\tc{\$command1 \& command2 \& command3 \&}{Все три команды выполняются фоном:}

Команды управления заданиями: \\
\begin{tabular}{|c|p{13cm}|}
\hline
\rowcolor{gray!40}
Команда & Значение \\
\hline
bg & Переместить текущее приостановленное задание в фоновый режим \\
bg \%n & Переместить приостановленное задание номер n в фоновый режим (пример: bg \%1) \\
fg & Переместить текущее фоновое задание на передний план \\
fg \%n & Переместить фоновое задание номер n на передний план (пример: fg \%2) \\
kill \%n & Завершить фоновое задание номер n (пример: kill \%3) \\
jobs & Просмотр списка заданий оболочки \\
\hline
\end{tabular}

Перевести команду в фон: нажать \texttt{Ctrl-Z}, а затем \texttt{bg}.

\subsection{Явные подоболочки}
\cn{\$(cd /usr/local \&\& ls)}
\cn{\$ cat package.tar.gz | $\backslash$}
\ci{(mkdir -p /tmp/other \&\& cd /tmp/other \&\& tar xzvf -)}
Копировать файлы в существующий каталог на другом хосте через SSH: \\
\cn{\$tar czf - dir1 | ssh myhost '(cd /tmp/dir2 \&\& tar xvf -)'}

\subsection{Замена процесса}
\tc{\$exec ls}{ls заменяет дочернюю оболочку, запускается и завершает работу:}



	\section{Создание дерзких однострочников}

Сдвинуть нумерацию файлов: \\
\cn{\$ paste <(echo {1..10}.jpg | sed 's/ /$\backslash$n/g') $\backslash$ }
\ci{\indent\indent <(echo {0..9}.jpg | sed 's/ /$\backslash$n/g') $\backslash$ }
\ci{| sed 's/\^{}/mv /' $\backslash$ }
\ci{| bash }

\noindent Найти 17-й символ: \\
\ci{\$echo \{A..Z\} | awk '\{print \$(17)\}'}
\ci{\$echo \{A..Z\} | sed 's/ //g' | cut -c17 }

\noindent Вывести названия месяцев: \\
\ci{\$echo 2021-\{01..12\}-01 | xargs -n1 date +\%B -d}

\noindent Количество символов в самом длинном имени файла: \\
\ci{\$ls | awk '\{print "echo -n"{}, \$0, "{}| wc -c"\}' | bash $\backslash$ }
\ci{\indent\indent | sort -nr | head -n1}

\noindent Проверка совпадающих пар файлов: \\
\ci{\$diff <(ls *.jpg | sed 's/$\backslash$.[\^{}.]*\$//') $\backslash$ }
\ci{\indent \indent<(ls *.txt | sed 's/$\backslash$.[\^{}.]*\$//') $\backslash$ }
\ci{\indent| grep '\^{}[<>]' $\backslash$ }
\ci{\indent| awk '/</\{print \$2 ".jpg"\} />/\{print \$2 ".txt"\}' }
\noindent Другой способ: \\
\ci{\$ls -1 \$( ls *.\{jpg,txt\} $\backslash$ }
\ci{\indent| sed 's/$\backslash$.[\^{}.]*\$//' $\backslash$ }
\ci{\indent| uniq -c $\backslash$ }
\ci{\indent | awk '/\^{} *1 /\{print \$2 "*"\}' ) }

\noindent Создание 1000 файлов со случайными именами и случайными словами: \\
\cn{\$ yes 'shuf -n \$RANDOM -o \$(pwgen -N1 10).txt /usr/share/dict/words' $\backslash$ }
\ci{\indent | head -n 1000 $\backslash$ }
\ci{\indent | bash}

	\section{Управление буфером обмена}
\noindent Копирование и вставка из первичного буфера: \\
\ci{\$echo -n | xclip}
\ci{\$xclip -o}

\noindent Копирование и вставка из системного буфера: \\
\ci{\$echo https://oreilly.com | xclip -selection clipboard}
\ci{\$xclip -selection clipboard -o}

	\section{Переменные окружения}
Получить все системные переменные окружения: \\
\ci{\$printenv | sort -i | less}
\ci{\$env}
\ccn{\$MY\_VARIABLE=10}{установить значение локальной переменной}
\ccn{\$export E="variable"}{установить переменную оболочки}


	\section{web}

\cn{\$wget}
\ccn{\$curl -li https://localhost}{содержимое сайта}

\noindent Парсинг страницы html по классам объектов: \\
\ci{\$curl -s https://efficientlinux.com/areacodes.html $\backslash$ }
\ci{\indent| hxnormalize -x $\backslash$}
\ci{\indent| hxselect -c -s@ '\#ac .ac, \#ac .state, \#ac .cities'}

\cn{\$lynx -dump https://efficientlinux.com/areacodes.htm}



\section{Неразобранное}
\noindent Полезные команды: \\
\cci{\$cat /etc/os-release}{информация о версии системы}
\cci{\$netstat -tulpen}{Открытые порты}
\cci{\$ps -uax}{все запущенные процессы для всех пользователей:}

\noindent Код возврата: \\
\ci{\$?}

\noindent Обработка файла построчно:\\
\ci{\$cat /etc/hosts | while read line; do}
\ci{\indent echo "\$line" | wc -c}
\ci{done}

	
\end{document}
