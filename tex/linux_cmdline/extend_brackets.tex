\section{Расширение команд с помощью фигурных скобок}

\ccn{\$echo \{1..10\}}{Вперед, начиная с 1: 1 2 3 4 5 6 7 8 9 10}
\ccn{\$echo \{10..1\}}{Назад, начиная с 10}
\ccn{\$echo \{01..10\}}{С ведущими нулями (для равной ширины)}
\ccn{\$echo \{1..1000..100\}}{Приращение сотнями, начиная с 1}
\ccn{\$echo \{1000..1..100\}}{Уменьшение сотнями, начиная с 1000}
\ccn{\$echo \{01..1000..100\}}{С ведущими нулями}

\noindent Фигурные скобки vs квадратные: \\
\cci{\$ls file[2-4]}{Соответствует существующим именам файлов}
\cci{\$ls file\{2..4\}}{Вычисляется в: file2 file3 file4}

\ccn{\$echo \{A..Z\}}{A B C \ldots X Y Z}
\ccn{\$echo \{A..Z\} | tr -d ' '}{Удалить пробелы: ABC\ldots XYZ}