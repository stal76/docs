\section{grep}

\ccn{\$grep his frost}{Вывести строки, содержащие «his»}
\ccn{\$grep -w his frost}{Искать точное соответствие «his»}
\ccn{\$grep -i his frost}{игнорировать регистр букв}
\ccn{\$grep -l his *}{В каком файле содержится «his»?}

\begin{tabular}{|p{5cm}|c|p{6cm}|}
\hline
\rowcolor{gray!40}
\centering Соответствие & \makecell{Используемые \\ выражения} & \makecell{Пример} \\
\hline
Начало строки & \texttt{\^{}} & 
\texttt{\^{}a} = строка, начинающаяся с a \\
\hline
Конец строки & \texttt{\$} & 
\makecell{\texttt{!\$} = строка, заканчивающаяся \\ восклицательным знаком} \\
\hline
Любой одиночный символ (кроме новой строки) & \texttt{.} &
\texttt{...} = любые три последовательных символа \\
\hline
Знаки вставки, доллара или любой другой специальный символ с &
\texttt{${\backslash}$c} &
\texttt{\$} = знак доллара \\
\hline
Ноль или более вхождений выражения E & \texttt{E*} &
\texttt{\_*} = ноль или более знаков подчеркивания \\
\hline
Любой одиночный символ в наборе & \texttt{[characters]} &
\texttt{[aeiouAEIOU]} = любая гласная \\
\hline

Любой одиночный символ, не входящий в набор &
\texttt{[\^{}characters]}  &
\texttt{[\^{}aeiouAEIOU]} = любая не гласная \\

\hline

Любой символ в диапазоне  между с1 и с2 &
\texttt{[c1-c2]} &
\texttt{[0-9]} = любая цифра \\

\hline

Любой символ вне диапазона между c1 и c2 &
\texttt{[\^{}c1-c2]} &
\texttt{[\^{}0-9]} = любой нецифровой символ \\

\hline

Любое из двух выражений E1 или E2 &
\makecell{\texttt{E1$\backslash$|E2} \\ для grep и sed, \\ \texttt{E1/E2} для awk} &
\texttt{one|two} = или one, или two \\

\hline

Группировка выражения E с учетом приоритета &
\makecell{\texttt{$\backslash$(E$\backslash$)} \\ для grep и sed,\\ \texttt{(E)} для awk} & 
\makecell[l]{\texttt{$\backslash$(one$\backslash$|two)$\backslash$*} \\ \texttt{(one|two)*} = ноль или более \\ вхождений one или two} \\

\hline

\end{tabular}
\\ 

\ccn{\$grep -v '\^{}\$' myfile}{все непустые строки, \texttt{-v} - исключает}
\ccn{\$grep 'cookie$\backslash$|cake' myfile}{содержащие либо cookie, либо cake}
\ccn{\$grep '<.*>' page.html}{< появляется перед символом >}

\ccn{\$grep -F w. frost}{отключить регулярные выражения}
\ccn{\$fgrep w. frost}{поиск без регулярных выражений}

\tc{\$grep -f <filename> ...}{Поиск по списку шаблонов из файла:}