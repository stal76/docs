\section{Архивирование}

По материалам \href{https://losst.pro/arhivatsiya-v-linux}{отсюда}

\subsection{tar}

\noindent Основные опции:\\
\indent\texttt{A} - добавить файл к архиву \\
\indent\texttt{c} - создать архив в linux \\
\indent\texttt{d} - сравнить файлы архива и распакованные файлы в файловой системе \\
\indent\texttt{j} - сжать архив с помощью Bzip \\
\indent\texttt{z} - сжать архив с помощью Gzip \\
\indent\texttt{r} - добавить файлы в конец архива \\
\indent\texttt{t} - показать содержимое архива \\
\indent\texttt{u} - обновить архив относительно файловой системы \\
\indent\texttt{x} - извлечь файлы из архива \\
\indent\texttt{v} - показать подробную информацию о процессе работы \\
\indent\texttt{f} - файл для записи архива \\
\indent\texttt{-C} - распаковать в указанную папку \\
\indent\texttt{--strip-components} - отбросить n вложенных папок \\


\tc{\$ tar -cvf archive.tar /path/to/files}{Создать архив:}
\tc{\$ tar -xvf archive.tar.gz}{Распаковать архив tar:}
\tc{\$ tar -zcvf archive.tar.gz /path/to/files}{Создать сжатый архив:}
\indent\texttt{\$ tar -zxvf archive.tar.gz}

\tc{\$ tar -rvf archive.tar file.txt}{Добавить файл в архив:}
\tc{\$ tar -xvf archive.tar file.txt}{Извлечение одного файла:}
\tc{\$ tar -xvf archive.tar --wildcards '*.php'}{Извлечь несколько файлов по шаблону:}
\tc{\$ tar -xvf archive.tar -C /path/to/dir}{Распаковать в нужную папку:}

\subsection{Gzip}

\noindent\texttt{\$ gzip опции файл} \\
\noindent\texttt{\$ gunzip опции файл} \\

\noindent Опции: \\
\indent\texttt{-c} - выводить архив в стандартный вывод \\
\indent\texttt{-d} - распаковать \\
\indent\texttt{-f} - принудительно распаковывать или сжимать \\
\indent\texttt{-l} - показать информацию об архиве \\
\indent\texttt{-r} - рекурсивно перебирать каталоги \\
\indent\texttt{-0} - минимальный уровень сжатия \\
\indent\texttt{-9} - максимальный уровень сжатия \\

\subsection{Zip}

\noindent\texttt{\$ zip опции файлы} \\
\noindent\texttt{\$ unzip опции архив} \\

\noindent Опции: \\
\indent\texttt{-d} - удалить файл из архива \\
\indent\texttt{-r} - рекурсивно обходить каталоги \\
\indent\texttt{-0} - только архивировать, без сжатия \\
\indent\texttt{-9} - наилучший степень сжатия \\
\indent\texttt{-F} - исправить zip файл \\
\indent\texttt{-e} - шифровать файлы
