\documentclass[12pt, a4paper]{article}
\usepackage[T1]{fontenc}
\usepackage[utf8]{inputenc}
\usepackage[english,russian]{babel}
\usepackage{amsmath}
\usepackage{amsfonts}
\usepackage{amssymb}
\usepackage{makeidx}
\usepackage{graphicx}
\begin{document}
	
{\Huge Git}

\section{Введение}

\subsection{Настройки}

Сохранить локальные настройки: \\
\indent \texttt{\$git config user.name <user\_name>} \\
\indent \texttt{\$git config user.email <user\_email>} \\

\noindent Варианты настроек: \\
\indent \texttt{\$git config {-}-system \ldots \indent \#} Системные: /etc/git/config \\
\indent \texttt{\$git config {-}-global \ldots \indent \#} Пользователя: ~/.gitconfig \\
\indent \texttt{\$git config {-}-local \ldots \indent \#}  Проекта (default):	<project>/.git/config \\

\noindent\texttt{\$git config {-}-list \indent \#} - все конфигурационные параметры \\
\texttt{\$git config {-}-unset user.name \indent \#} - сбросить параметр \\ 
\texttt{\$git config {-}-remove-section user \indent \#} - удалить секцию \\ 
\texttt{\$git config {-}-global core.editor \ldots \indent \#} редактор для commit \\
	
\noindent Алиасы, пример: \\
\indent\texttt{\$git config {-}-global alias.c 'config {-}-global'} \\
\indent\texttt{\$git c {-}-list}


\subsection{Простые команды}

\texttt{\$git init \indent \#} - инициализация каталога проекта \\
\texttt{\$git status \indent \#} - текущий статус \\
\texttt{\$git add <file> \indent \#} - добавить файл в индекс \\
Сделать commit:\\
\indent\texttt{\$git commit \indent \#} - запустится редактор для ввода комментария \\
\indent\texttt{\$git commit -m <comment>} \\
\texttt{\$git add {-}- chmod=+x <file> \indent \#} - сделать файл исполняемым \\
	
\noindent Специальные файлы:\\
\indent.gitignore \indent - содержит список игнорируемых файлов \\
\indent.gitkeep \indent - в пустых каталогах, тогда они добавляются в индекс\\

\noindent\texttt{\$git reset HEAD <file> \indent \#} - убрать изменения в файле \\
\texttt{\$git  add -f <file> \indent \#} - добавить файл из игнорируемого каталога\\
	

\texttt{\$git add -p <file> \indent \#} - с вопросом по каждому изменению \\

\noindent Добавление всех изменений сразу в commit:\\
\indent \texttt{\$git commit -all \indent} \\
\indent \texttt{\$git commit -am <comment>} \\
\indent \texttt{\$git commit -m <comment> file1 file2 \ldots" \#} - несколько файлов \\

\noindent Удалить файл: \\
\indent \texttt{\$git rm <file> \indent \#} - удаление из каталога и индекса \\
\indent \texttt{\$git rm -r <file> \indent \#} - удаление из каталога\\
\indent \texttt{\$git rm {-}-cached <file> \indent \#} - удаление из индекса \\
\indent \texttt{\$git rm -f <file> \indent \#} - удаление с несохраненными изменениями\\

\noindent Переименовать файл:\\
\indent \texttt{\$git mv <old\_name> <new\_name>}

\section{Удаление изменений}

\texttt{\$git checkout -f \indent \#} - удалить все изменения \\
\texttt{\$git checkout -f file \indent \#} - удалить изменения в файле \\
\texttt{\$git reset --hard \indent \#}  - сброс изменений \\
\texttt{\$git clean -dxf \indent \#} - очистка от изменений \\
\indent -d - директории \\
\indent -x - файлы которые не отслеживаются \\


\section{Ветки}

Работа с ветками: \\
\indent \texttt{\$git branch -v \indent \#} - список веток с краткой информацией\\
\indent \texttt{\$git branch <name> \indent \#} - создать новую ветку \\
\indent \texttt{\$git checkout <name> \indent \#} - переключиться на ветку\\
\indent \texttt{\$git checkout -b <name> \indent \#} = create + checkout \\

\noindent Изменения в ветках: \\

\indent \texttt{\$git stash \indent \#} - убрать все изменения и запомнить \\
\indent \texttt{\$git stash pop \indent \#} - вернуть запомненные изменения \\

\noindent Сделать новую ветку указывающую на commit:\\
\indent \texttt{\$git branch <name> <commit>}  \\
Передвинуть ветку на commit (или другую ветку):  \\
\indent \texttt{\$git branch -f <name> <commit|branch>} \\ 
\indent \texttt{\$git checkout -B <name> <commit> \indent \#}  = \texttt{git branch -f}  + \texttt{git checkout} \\
Передвинуться на commit:\\
\indent \texttt{\$git checkout <commit>} \\


\noindent Cкопировать commit на текущую ветку: \\
\indent \texttt{\$git cherry-pick <commit>}   \\

\noindent Восстановить версии файлов из commit:\\
\indent \texttt{\$git checkout <commit> <filenames>} \\
Сбросить индекс для файла: \\
\indent \texttt{\$git reset <filename>} \\
Вернуть файл из repository: \\
\indent \texttt{\$git checkout HEAD <filename>}  \\
Вернуть файл из индекса: \\
\indent \texttt{\$git checkout <filename>}  \\

\noindent Просмотр истории: \\
\indent \texttt{\$git log {-}-oneline} \\
\indent \texttt{\$git log <branch> {-}-oneline} \\
\indent \texttt{\$git show <commit>} \\
\indent \texttt{\$git show HEAD\~  \indent \#} - родитель commit \\
\indent \indent \texttt{HEAD\~{}\~{}\~{} = HEAD\~{}3}  \\
\indent \indent \texttt{HEAD = @}  \\

\noindent Показать содержимое файла из commit: \\
\indent \texttt{\$git show <commit>:<filename>} \\
\noindent Найти commit по строке: \\
\indent \texttt{\$git show <commit>:/<string>} \\

\noindent Выполнить merge master и branch, используется Fast-Forward: \\
\indent \texttt{\$git merge <branch>} \\

\noindent Продвинуть ветку: \\
\indent \texttt{\$git branch -f <branch> <commit> \indent \#} - на commit   \\
\indent \texttt{\$git branch -f <branch> ORIG\_HEAD \indent \#} - на предыдующий HEAD  \\

\noindent Удаление ветки: \\
\indent \texttt{\$git branch -d <branch> \indent \#} -только если объединение с текущей \\
\indent \texttt{\$git branch -D <branch> \indent \#} - если в ней есть отличные commit \\

\noindent Восстановить удаленную ветку: \\
\indent \texttt{\$git branch <branch> <commit>} \\


\section{Reference logs}

\texttt{\$cat .git/logs/HEAD} \\
\texttt{\$git reflog} \\
\texttt{\$git reflog <branch>} \\
\texttt{\$git reflog} \indent  алиас для \indent \texttt{\$git log {-}-oneline -g} \\
\texttt{\$git reflog {-}-date=iso} \\

\section{XXXXXXXXXXXXXXXXXXXXXXXX}



\section{Отмена commit}
\texttt{\$git revert <commit> \#} - создает commit отменяющий изменения в <commit> \\
\texttt{\$git revert <commit> -m 1 \indent \#}  - отмена commit слияния \\

	
\end{document}
