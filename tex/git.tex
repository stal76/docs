\documentclass[12pt, a4paper]{article}
\usepackage[utf8]{inputenc}
\usepackage[russian]{babel}
\usepackage{cmap}
\usepackage{amsmath}
\usepackage{amsfonts}
\usepackage{amssymb}
\usepackage{makeidx}
\usepackage{graphicx}
\usepackage{hyperref}

\begin{document}
		
\begin{center} {\Huge Git - справка} \end{center}

Составлено по материалам 
\href{https://www.youtube.com/watch?v=W4hoc24K93E&list=PLDyvV36pndZFHXjXuwA_NywNrVQO0aQqb}{курса} от JavaScript.ru \\


\section{Введение}

\subsection{Настройки}

Сохранить локальные настройки: \\
\indent \texttt{\$git config user.name <user\_name>} \\
\indent \texttt{\$git config user.email <user\_email>} \\

\noindent Варианты настроек: \\
\indent \texttt{\$git config {-}-system \ldots \indent \#} Системные: /etc/git/config \\
\indent \texttt{\$git config {-}-global \ldots \indent \#} Пользователя: ~/.gitconfig \\
\indent \texttt{\$git config {-}-local \ldots \indent \#}  Проекта (default):	<project>/.git/config \\

\noindent\texttt{\$git config {-}-list \indent \#} - все конфигурационные параметры \\
\texttt{\$git config {-}-unset user.name \indent \#} - сбросить параметр \\ 
\texttt{\$git config {-}-remove-section user \indent \#} - удалить секцию \\ 
\texttt{\$git config {-}-global core.editor \ldots \indent \#} редактор для commit \\
	
\noindent Алиасы, пример: \\
\indent\texttt{\$git config {-}-global alias.c 'config {-}-global'} \\
\indent\texttt{\$git c {-}-list}


\subsection{Простые команды}

\texttt{\$git init \indent \#} - инициализация каталога проекта \\
\texttt{\$git status \indent \#} - текущий статус \\
\texttt{\$git add <file> \indent \#} - добавить файл в индекс \\
Сделать commit:\\
\indent\texttt{\$git commit \indent \#} - запустится редактор для ввода комментария \\
\indent\texttt{\$git commit -m <comment>} \\
\texttt{\$git add {-}- chmod=+x <file> \indent \#} - сделать файл исполняемым \\
	
\noindent Специальные файлы:\\
\indent.gitignore \indent - содержит список игнорируемых файлов \\
\indent.gitkeep \indent - в пустых каталогах, тогда они добавляются в индекс\\

\noindent\texttt{\$git reset HEAD <file> \indent \#} - убрать изменения в файле \\
\texttt{\$git  add -f <file> \indent \#} - добавить файл из игнорируемого каталога\\
	

\texttt{\$git add -p <file> \indent \#} - с вопросом по каждому изменению \\

\noindent Добавление всех изменений сразу в commit:\\
\indent \texttt{\$git commit -all \indent} \\
\indent \texttt{\$git commit -am <comment>} \\
\indent \texttt{\$git commit -m <comment> file1 file2 \ldots" \#} - несколько файлов \\

\noindent Удалить файл: \\
\indent \texttt{\$git rm <file> \indent \#} - удаление из каталога и индекса \\
\indent \texttt{\$git rm -r <file> \indent \#} - удаление из каталога\\
\indent \texttt{\$git rm {-}-cached <file> \indent \#} - удаление из индекса \\
\indent \texttt{\$git rm -f <file> \indent \#} - удаление с несохраненными изменениями\\

\noindent Переименовать файл:\\
\indent \texttt{\$git mv <old\_name> <new\_name>}

\section{Удаление изменений}

\texttt{\$git checkout -f \indent \#} - удалить все изменения \\
\texttt{\$git checkout -f file \indent \#} - удалить изменения в файле \\
\texttt{\$git reset --hard \indent \#}  - сброс изменений \\
\texttt{\$git clean -dxf \indent \#} - очистка от изменений \\
\indent -d - директории \\
\indent -x - файлы которые не отслеживаются \\


\section{Ветки}

Работа с ветками: \\
\indent \texttt{\$git branch -v \indent \#} - список веток с краткой информацией\\
\indent \texttt{\$git branch <name> \indent \#} - создать новую ветку \\
\indent \texttt{\$git checkout <name> \indent \#} - переключиться на ветку\\
\indent \texttt{\$git checkout -b <name> \indent \#} = create + checkout \\

\noindent Изменения в ветках: \\

\indent \texttt{\$git stash \indent \#} - убрать все изменения и запомнить \\
\indent \texttt{\$git stash pop \indent \#} - вернуть запомненные изменения \\

\noindent Сделать новую ветку указывающую на commit:\\
\indent \texttt{\$git branch <name> <commit>}  \\
Передвинуть ветку на commit (или другую ветку):  \\
\indent \texttt{\$git branch -f <name> <commit|branch>} \\ 
\indent \texttt{\$git checkout -B <name> <commit> \indent \#}  = \texttt{git branch -f}  + \texttt{git checkout} \\
Передвинуться на commit:\\
\indent \texttt{\$git checkout <commit>} \\


\noindent Cкопировать commit на текущую ветку: \\
\indent \texttt{\$git cherry-pick <commit>}   \\

\noindent Восстановить версии файлов из commit:\\
\indent \texttt{\$git checkout <commit> <filenames>} \\
Сбросить индекс для файла: \\
\indent \texttt{\$git reset <file>} \\
Вернуть файл из repository: \\
\indent \texttt{\$git checkout HEAD <file>}  \\
Вернуть файл из индекса: \\
\indent \texttt{\$git checkout <file>}  \\

\noindent Просмотр истории: \\
\indent \texttt{\$git log {-}-oneline} \\
\indent \texttt{\$git log <branch> {-}-oneline} \\
\indent \texttt{\$git show <commit>} \\
\indent \texttt{\$git show HEAD\~  \indent \#} - родитель commit \\
\indent \indent \texttt{HEAD\~{}\~{}\~{} = HEAD\~{}3}  \\
\indent \indent \texttt{HEAD = @}  \\

\noindent Показать содержимое файла из commit: \\
\indent \texttt{\$git show <commit>:<file>} \\
\noindent Найти commit по строке: \\
\indent \texttt{\$git show <commit>:/<string>} \\

\noindent Выполнить merge master и branch, используется Fast-Forward: \\
\indent \texttt{\$git merge <branch>} \\

\noindent Продвинуть ветку: \\
\indent \texttt{\$git branch -f <branch> <commit> \indent \#} - на commit   \\
\indent \texttt{\$git branch -f <branch> ORIG\_HEAD \indent \#} - на предыдущий HEAD  \\

\noindent Удаление ветки: \\
\indent \texttt{\$git branch -d <branch> \indent \#} -только если объединение с текущей \\
\indent \texttt{\$git branch -D <branch> \indent \#} - если в ней есть отличные commit \\

\noindent Восстановить удаленную ветку: \\
\indent \texttt{\$git branch <branch> <commit>} \\


\section{Reference logs}

\texttt{\$cat .git/logs/HEAD} \\
\texttt{\$git reflog} \\
\texttt{\$git reflog <branch>} \\
\texttt{\$git reflog} \indent  алиас для \indent \texttt{\$git log {-}-oneline -g} \\
\texttt{\$git reflog {-}-date=iso} \\


\section{Reset}

\noindent Сброс (по умолчанию mix): \\
\indent \texttt{\$git reset {-}-hard <commit> \indent\#} - Жесткий сброс \\
\indent \texttt{\$git reset {-}-mix <commit> \indent\#} - Смешанный сброс \\
\indent \texttt{\$git reset {-}-soft <commit> \indent\#} - Мягкий сброс \\
\indent \indent 	\texttt{@ \indent} - последний commit \\
\indent \indent 	\texttt{@\~ \indent} - предыдущий commit \\

Отличия видов reset (из \texttt{\$git reset {-}{-}help}): \\

\texttt {
\begin{tabular}{c c c c l c c c}
	working & index & HEAD & target &  & working & index & HEAD \\
	\hline
	A & B & C & D & {-}{-}soft & A & B & D \\
	 &  & & & {-}{-}mixed & A & D & D \\
	 &  & & & {-}{-}hard & D & D & D \\
\end{tabular}
}


\noindent \texttt{\$git reflog master} \\

\noindent Взять комментарий из другого commit: \\
\indent \texttt{\$git commit -c <commit> \indent\#} - откроется редактор  \\
\indent \texttt{\$git commit -C <commit> \indent\#} - не будет открываться редактор \\
\indent \indent \texttt{ {-}-reset-author \indent} - сбросить автора (по умолчанию из commit) \\

\noindent \texttt{\$git commit {-}-amend \indent\#}  - исправить в последнем commit \\
\indent \texttt{ {-}-no-edit\indent\#} - не открывать редактор \\ 
\indent \texttt{ {-}-reset-author} \\

\noindent Очистка индекса от всех изменений: \\
\indent \texttt{\$git reset HEAD} \\
\noindent Удалить изменения по файлу в индексе \\
\indent \texttt{\$git reset <file>} \\
\noindent Поместить файл из commit в индекс (в директории не меняется):\\
\indent \texttt{\$git reset <commit> <file>} \\


\section{Просмотр изменений}

\noindent \texttt{\$git diff <commit1> <commit2>} \\
\noindent \texttt{\$git diff <commit1>..<commit2>} \\

\noindent Что изменилось в branch2 с момента отделение от ветки branch1: \\
\indent \texttt{\$git diff <branch1>...<branch2>} \\
\noindent Изменения в директории с момента последнего commit: \\
\indent \texttt{\$git diff HEAD} \\
\noindent \texttt{\$git diff} - Изменения по сравнению с индексом \\
\noindent \texttt{\$git diff {-}-cached} - Изменения индекса с последним commit \\

\noindent При commit в редакторе показывать изменения: \\
\indent \texttt{\$git commit -v} \\
\noindent Включить автоматическое отображение изменений при commit: \\
\indent \texttt{\$git config {-}-global commit.verbose true} \\

\noindent \texttt{\$git diff <file>} - Изменения в файле \\
\noindent Только имена изменившихся файлов: \\
\indent \texttt{\$git diff {-}-name-only <commit1> <commit2>}\\
\noindent Сравнить разные файлы из разных commit: \\
\indent \texttt{\$git diff <commit1>:<file1> <commit2>:<file2>} \\
\noindent Сравнить 2 произвольных файла без привязки к git: \\
\indent \texttt{\$git diff {-}-no-index <file1> <file2>} \\



\section{Просмотр истории}

\noindent \texttt{\$git log \indent\#} - Вывод истории commit достижимых из HEAD \\
\noindent \texttt{\$git log {-}-pretty=<value>} \\
\indent \texttt{medium (default)} \\
\indent\texttt{oneline} \\
\indent\texttt{format:'\%h \%cd | \%s\%d [\%an]' }  - id, date | comment ссылки автор \\
\indent\indent\texttt{\%C(<colorname)\indent} - задать цвет \\


	\indent \texttt{{-}-abbrev-commit \indent\#} - обрезать id commit \\
\noindent \texttt{\$git log {-}-oneline}  = \texttt{\$git log {-}-pretty=oneline {-}-abbrev-commit} \\
\indent \texttt{{-}-decorate-short} \\
\indent \texttt{{-}-date=short \indent\#} - только даты \\
\indent \texttt{{-}-date=format:'\%F \%R' \indent\#}  - берет флаги из strftime \\

\noindent \texttt{\$git log <branch> \indent\#}  - log по commit достижимым из ветки \\
\indent \texttt{{-}-graph \indent\#} - в виде дерева \\
\indent \texttt{{-}-all \indent\#} - из всех commit \\

\noindent \texttt{\$git log gui} \\
\noindent \texttt{\$git log feature {$ \mathbin{\char`\^} $}master \indent\#} - log по feature кроме master \\
\noindent \texttt{\$git log master..feature} \\
\indent \texttt{{-}-boundary \indent\#} - включая пограничный commit \\


\noindent \texttt{\$git log ..feature} - симметрическая разница \\

\noindent \texttt{\$git log <file>} - история по файлу \\
\indent \texttt{-p} - показывать различия \\
\indent \texttt{{-}-follow} - если был переименован, то и по старым названиям \\

\noindent История между commit по списку файлов: \\
\indent \texttt{\$git log <commit1>..<commit2> <file1> <file2> ...} \\

\noindent Поиск по строке по commit достижимым из HEAD: \\
\indent \texttt{\$git log {-}-grep <string>}\\
\noindent \texttt{\$git log {-}-grep <str1> {-}-grep <str2> \#} - str1 или str2 в сообщении \\
\indent \indent \texttt{{-}-all-match \indent} - по всем совпадение \\
\indent \indent \texttt{-p \indent} - Perl type совместимые регулярные выражения \\
\indent \indent \texttt{-F \indent} - отключить регулярные выражения \\
\indent \indent \texttt{-i \indent} - регистро-независимый поиск \\

\noindent Поиск по содержимому: \\
\indent \texttt{\$git log -G <string>}\\
\noindent Все изменения со строки i1 по i2 в <file>: \\
\indent \texttt{\$git log -L i1,i2:<file>}\\
\noindent Регулярные выражения для начала и конца фрагмента: \\
\indent \texttt{\$git log -L r1,r2:<file> - r1,r2}\\
\noindent Регулярное выражение <func> для имени функции: \\
\indent \texttt{\$git log :<func>:<file>} \\

\noindent \texttt{\$git log}: \\
\indent \texttt{{-}-author} \\
\indent \texttt{{-}-commiter} \\
\indent \texttt{{-}-before 'before 3 months'} \\
\indent \texttt{{-}-after} \\

\noindent \texttt{\$git blame <file> \indent\#} - кто менял строки файла \\



\section{Слияние}

\noindent \texttt{\$git merge <branch> \indent\#} - объединить ветки \\

\noindent При конфликте взять версию файла: \\
\indent \texttt{\$git checkout {-}-ours <file> \indent\#} - из текущей ветки  \\
\indent \texttt{\$git checkout {-}-theirs <file> \indent\#} - из объединяемой ветки \\

\noindent \texttt{\$git checkout {-}-merge} \\

\noindent Вернуть в состоянии до merge: \\
\indent \texttt{\$git reset {-}-hard} \\
\noindent Оставить которые хорошо объединились, проблемные вернуть в начальное состояние: \\
\indent \texttt{\$git reset {-}-merge} \\

\noindent \texttt{\$git checkout {-}-conflict-diff3 {-}-merge <file>} \\

\noindent После окончания разрешения конфликтов создает окончательный commit слияния: \\
\indent \texttt{\$git merge {-}-continue}

\noindent id сливаемого commit в .git/MERGE\_HEAD \\

\noindent Отмена слияния: \\
\indent \texttt{\$git reset {-}-hard {@}\~} \\

\noindent Слияние без автоматического commit: \\
\indent \texttt{\$git merge <branch> {-}-no-commit} \\

\noindent Слияние без Fast Forward (перемотки), создается новый commit: \\
\indent \texttt{\$git merge {-}-no-ff {-}-no-edit <commit>} \\

\noindent Слияние без сохранения истории сливаемого commit: \\
\indent \texttt{\$git merge {-}-squash <commit>} \\





\section{Копирование commit}

\texttt{\$git cherry-pick <commit> \indent\#} - скопировать на текущую ветку \\
\indent\texttt{-x} - добавить строку с информацией откуда скопирован commit \\

\noindent Скопировать все commit из <commit2> в <commit1> с момента разделения: \\
\indent\texttt{\$git cherry-pick <commit1>..<commit2>}\\

\noindent При возникновении конфликта: \\
\indent\texttt{\$git cherry-pick {-}-abort \indent\#} - отмена \\
\indent\texttt{\$git cherry-pick {-}-continue \indent\#} - продолжить \\
\indent\texttt{\$git cherry-pick {-}-quit \indent\#} - оставить только успешно скопированные \\

\noindent Скопировать без создания commit \\
\indent\texttt{\$git cherry-pick {-}-no-commit <commit>} \\



\section{Перенос веток}

Чтобы текущая ветка начиналась с последнего commit в <commit>: \\
\indent\texttt{\$git rebase <commit>} \\
Полностью отказаться, если были неразрешенные конфликты: \\
\indent\texttt{\$git rebase {-}-abort} \\
Остаться в текущем незавершенном состоянии: \\
\indent\texttt{\$git rebase {-}-quit}  \\
Пропустить все изменения из текущего проблемного commit и продолжить дальше: \\
\indent\texttt{\$git rebase {-}-skip } \\


\noindent Продолжить после разрешение конфликта: \\
\indent\texttt{\$git rebase {-}-continue } \\
Узнать id ветки до предыдущего состояния: \\ 
\indent\texttt{\$cat .git/ORIG\_HEAD } \\
Просмотр изменений по ветке: \\
\indent\texttt{\$git reflog <branch> -1}  \\
Просмотр информации по последнему commit: \\
\indent\texttt{\$git show {-}-quiet <branch>{@}\{1\}} \\ 
Вернуться на состояние последнего commit: \\
\indent\texttt{\$git reset {-}-hard <commit>{@}\{1\}} \\ 


\noindent\texttt{\$git rebase <commit1> <commit2>} = \\
\indent\indent \texttt{\$git checkout <commit2>} + \texttt{\$git rebase <commit1>} \\

\noindent Для каждого переносимого commit запустить команду (обычно тест), при ошибке произойдет откат изменения: \\
\indent\texttt{\$git rebase -x <command> <branch>} \\
\noindent После исправления ошибок: \\
\indent\texttt{\$git add \ldots} \\
\indent\texttt{\$git commit {-}-amend {-}-no-edit} \\

\noindent Перенос всех commit в ветку branch из текущей ветки начиная с <commit>: \\
\indent \texttt{\$git rebase {-}-onto <branch> <commit>}  \\

\noindent Скопировать 2 последних commit: \\
\indent \texttt{\$git cherry-pick master~2..master \indent} \\
\noindent В master откатиться на commit назад: \\
\indent \texttt{\$git branch -f master master~2} \\

\noindent Если делали в ветке commit слияния, скопировать commit кроме commit слияния: \\
\indent \texttt{\$git rebase {-}-rebase-merges <commit>} \\

\noindent Интерактивное преобразование commit начиная с <commit>: \\
\indent \texttt{\$git rebase -i <commit>}  \\
\noindent Просмотр списка commit для преобразования: \\
\indent \texttt{\$git rebase {-}-edit-todo}  \\

\noindent \texttt{\$git rebase {-}-continue} \\
\noindent \texttt{\$git rebase -i @~3} - 3 commit назад \\
\noindent \texttt{\$git rebase {-}-no-ff}  - (no Fast Forward) - копировать все commit, а не только с первого измененного \\

\noindent Внести исправления в commit: \\
\indent \texttt{\$git commit -a {-}-fixup=<commit>}   - можно @~ - предыдущий \\
\indent \texttt{\$git rebase -i {-}-autosquash} - сольет fixup по commit \\





\section{Отмена commit}
\texttt{\$git revert <commit> \#} - создает commit отменяющий изменения в <commit> \\
\texttt{\$git revert <commit> -m 1 \indent \#}  - отмена commit слияния \\

	
\end{document}
