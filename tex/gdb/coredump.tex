\section{Coredump}

\noindent Запуск с coredump: \\
\indent \texttt{gdb -c <core> <exe>} \\

\noindent Просмотр настроек размера coredump (нужен параметр \texttt{core file size}): \\
\indent \texttt{ulimit -a }\\
\noindent Установить неограниченный размер (после перезагрузки сбрасывается): \\
\indent \texttt{ulimit -c unlimited} \\

\noindent Шаблон создания coredump: \\
\indent \texttt{/proc/sys/kernel/core\_pattern} \\
\noindent Пример шаблона для создания в текущем каталоге: \\
\indent \texttt{echo "core.\%e.\%p" | sudo tee /proc/sys/kernel/core\_pattern} \\


\noindent Может потребоваться сделать настройки для сервиса apport в\\ \texttt{~/.config/apport/settings}: \\
\indent \texttt{[main]} \\
\indent \texttt{unpackaged=true} \\

\noindent Для создания coredump может потребоваться запуск сервиса apport: \\
\indent \texttt{sudo service apport start}

\noindent Дампы могут создаваться тут: \\
\indent \texttt{/var/crash} \\

\noindent Статья по настройке coredump: \\
\href{https://askubuntu.com/questions/1349047/where-do-i-find-core-dump-files-and-how-do-i-view-and-analyze-the-backtrace-st}{Where do I find core dump files, and how do I view and analyze the backtrace (stack trace) in one?}

