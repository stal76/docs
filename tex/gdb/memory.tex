\section{Просмотр содержимого памяти}

	\texttt{x/nfu address}

\indent {\bf n} - количество единиц (1 можно не указывать)

\indent {\bf f} - формат:
\begin{itemize}
	\setlength{\itemsep}{-1pt}
	\setlength{\parskip}{-1pt}
	\item {\bf o} - восьмиричный;
	\item {\bf x} - шестнадцатиричный;
	\item {\bf d} - десятичный;
	\item {\bf f} - число с плавающей запятой;
	\item {\bf i} - инструкция процессора;
	\item {\bf с} - символ;
	\item {\bf s} - строка.
\end{itemize}

\indent {\bf u} - единица данных:
\begin{itemize}
	\setlength{\itemsep}{-1pt}
	\setlength{\parskip}{-1pt}
	\item {\bf b} - байт;
	\item {\bf h} - полуслово (два байта);
	\item {\bf w} - слово (четыре байта);
	\item {\bf g} - восемь байт;
\end{itemize}
