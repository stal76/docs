\section{Точки останова}

Установка точек останова:
\begin{itemize}
	\setlength{\itemsep}{0pt}
	\setlength{\parskip}{0pt}
	\item \texttt{b[reak]} - на текущей строке
	\item \texttt{b[reak] [filename:]n} - в файле на строке n
	\item \texttt{b [filename:]function}
	\item \texttt{b *function+offset}
	\item \texttt{b *address}
	\item \texttt{f +-n} - на n строк ниже (выше)
	\item \texttt{break arg if condition} - точка останова с условием
	\item \texttt{condition n} newcondition
	\item \texttt{tbreak} - разово установить точку останова
	\item \texttt{i[nfo] breakpoints} - информация обо всех точка останова
	\item \texttt{disable [breakpoints] [n-m]} - деактивировать точки останова
	\item \texttt{enable [breakpoints] [n-m]} - активировать точки останова
\end{itemize}

Удаление точек останова:
\begin{itemize}
	\setlength{\itemsep}{0pt}
	\setlength{\parskip}{0pt}
	\item \texttt{clear} - удалить в текущей строке
	\item \texttt{clear [filename:]n}
	\item \texttt{clear [filename:]function}
	\item \texttt{d[elete]} [breakpoints] [n-m]
	\item \texttt{d[elete]} - удалить все точки останова
\end{itemize}

\section{Точки наблюдения}
\begin{itemize}
	\setlength{\itemsep}{0pt}
	\setlength{\parskip}{0pt}
	\item \texttt{wa[tch] expression} - остановить при изменении
	\item \texttt{rw[atch] expression} - остановить при чтении
	\item \texttt{aw[atch] expression} - остановить при чтении или изменении
	\item \texttt{i[nfo] watchpoints} - вывести информацию обо всех точках наблюдения
\end{itemize}

\section{Точки перехвата (catchpoins)}

\noindent Перехват C++ исключений: \\
\indent \texttt{catch [re]throw [regex] } \\
\indent \texttt{catch catch [regex]} \\
\noindent Вызов системных функций: \\
\indent \texttt{catch syscall write} \\
\noindent Загрузка/выгрузка .so файлов: \\
\indent \texttt{catch [un]load [regex]} \\
\noindent Перехват сигналов: \\
\indent \texttt{catch signal <signal>}
